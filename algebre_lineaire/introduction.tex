\chapter*{Introduction}
\addcontentsline{toc}{chapter}{Introduction}
\markboth{Introduction}{Introduction}

L'algèbre linéaire a le chic de surgir là où on ne l'attend pas
nécessairement. On croit qu'en étudiant l'actuariat les notions de
vecteurs et de matrices nous demeurerons étrangères. Et pourtant.

Les matrices et leur algèbre permettent de représenter, de traiter et
de résoudre efficacement de grands systèmes d'équations linéaires ou
d'équations différentielles. La notion d'erreur quadratique moyenne
s'apparente à la projection d'un vecteur dans un espace vectoriel. Les
notions d'indépendance stochastique et d'orthogonalité de vecteurs
sont liées. On décrit le comportement d'une chaîne de Markov à l'aide
d'une matrice de transition. Une classe de lois de probabilités
requiert de calculer l'exponentielle d'une matrice. Ce ne sont là que
quelques exemples où l'algèbre linéaire joue un rôle en théorie des
probabilités, en inférence statistique, en finance ou en théorie du
risque.

Le chapitre~\ref{chap:revision} revient sur les principales notions
d'algèbre linéaire normalement étudiées au collège et qu'il est
important de maîtriser dans ses études de premier cycle universitaire.
Nous nous concentrons sur l'établissement de liens entre des éléments
qui peuvent au premier chef sembler disparates. Le
chapitre~\ref{chap:valeurspropres} construit sur le précédent pour
introduire les concepts de valeurs propres, de vecteurs propres et de
diagonalisation d'une matrice. Ceux-ci jouent un rôle, entre autres,
en finance mathématique. Pour terminer sur des considérations
numériques ayant jusque là traversé le cours, le
chapitre~\ref{chap:decomposition} expose et compare très succinctement
différentes stratégies utilisées pour résoudre des systèmes
d'équations linéaires à l'aide d'un ordinateur.

Cette dernière partie du cours diffère passablement de celles qui
l'ont précédée, autant dans sa nature que dans le format des activités
d'apprentissage. La matière y est beaucoup plus mathématique et
abstraite, il y a de nombreux exercices et les considérations
informatiques sont réduites au minimum. Les
chapitres~\ref{chap:revision} et \ref{chap:valeurspropres} ne
comportent que de courtes sections faisant la démonstration de
fonctions R dédiées à l'algèbre linéaire.

Comme toujours, chaque chapitre comporte des d'exercices, avec les
réponses en fin de chapitre et les solutions complètes en annexe.

Je tiens à souligner la précieuse collaboration de MM.~Mathieu
Boudreault, Sébastien Auclair et Louis-Philippe Pouliot lors de la
rédaction des exercices et des solutions. Je remercie également
Mmes~Marie-Pier Laliberté et Véronique Tardif pour l'infographie des
pages couvertures.

%%% Local Variables:
%%% mode: latex
%%% TeX-master: "methodes_numeriques-partie_4"
%%% End:
