\section{Exercices}
\label{chap:valeurspropres:exercices}

%%%
%%% Fichers de solutions et de réponses
%%%

\Opensolutionfile{reponses}[reponses-valeurs_propres]
\Opensolutionfile{solutions}[solutions-valeurs_propres]

\begin{Filesave}{reponses}
\bigskip
\section*{Réponses}

\end{Filesave}

\begin{Filesave}{solutions}
\section*{Chapitre \ref{chap:valeurspropres}}
\addcontentsline{toc}{section}{Chapitre \protect\ref{chap:valeurspropres}}

\end{Filesave}

%%%
%%% Début des exercices
%%%


\begin{exercice}
  \label{ex:valeurspropres:base}
  Trouver l'équation caractéristique, les valeurs propres et les bases
  de vecteurs propres des matrices suivantes. Vérifier les réponses
  obtenues à l'aide de la fonction \texttt{eigen} de R.
  \begin{enumerate}
    \begin{multicols}{2}
    \item $\begin{bmatrix} 3 & 0 \\ 8 & -1 \end{bmatrix}$
    \item $\begin{bmatrix} 10 & -9 \\ 4 & -2 \end{bmatrix}$
    \end{multicols}
    \begin{multicols}{2}
    \item $\begin{bmatrix}
        4 & 0 & 1 \\
        -2 & 1 & 0 \\
        -2 & 0 & 1
      \end{bmatrix}$
    \item $\begin{bmatrix}
        5 &  6 &  2 \\
        0 & -1 & -8 \\
        1 &  0 & -2
      \end{bmatrix}$
    \end{multicols}
    \begin{multicols}{2}
    \item $\begin{bmatrix}
        0 & 0 &  2 & 0 \\
        1 & 0 &  1 & 0 \\
        0 & 1 & -2 & 0 \\
        0 & 0 &  0 & 1
      \end{bmatrix}$
    \end{multicols}
  \end{enumerate}
  \begin{sol}
    Dans tous les cas, la matrice mentionnée dans l'énoncé est notée
    $\mat{A}$.
    \begin{enumerate}
    \item On a
      \begin{align*}
        \det(\lambda \mat{I} - \mat{A})
        &=
        \begin{vmatrix}
          \lambda - 3 & 0 \\
          -8 & \lambda + 1
        \end{vmatrix} \\
        &= (\lambda - 3)(\lambda + 1).
      \end{align*}
      Les valeurs propres sont donc $\lambda_1 = 3$ et $\lambda_2 =
      -1$. D'une part, la forme échelonnée du système d'équations $(3
      \mat{I} - \mat{A}) \mat{x} = \mat{0}$ est
      \begin{displaymath}
        \begin{bmatrix} 0 & 0 \\ 1 & -\frac{1}{2} \end{bmatrix}
        \begin{bmatrix} x_1 \\ x_2 \end{bmatrix} =
        \begin{bmatrix} 0 \\ 0 \end{bmatrix},
      \end{displaymath}
      soit $x_1 = s/2$ et $x_2 = s$. Une base de vecteurs propres
      correspondant à $\lambda = 3$ est donc $(\frac{1}{2}, 1)$.
      D'autre part, le système d'équations $(-\mat{I} - \mat{A})
      \mat{x} = \mat{0}$ se réduit à
      \begin{displaymath}
        \begin{bmatrix} 1 & 0 \\ 0 & 0 \end{bmatrix}
        \begin{bmatrix} x_1 \\ x_2 \end{bmatrix} =
        \begin{bmatrix} 0 \\ 0 \end{bmatrix},
      \end{displaymath}
      soit $x_1 = 0$ et $x_2 = s$. Une base de vecteurs propres
      correspondant à $\lambda = -1$ est donc $(0, 1)$. Vérification:
\begin{Schunk}
\begin{Sinput}
> m <- matrix(c(3, 8, 0, -1), nrow = 2)
> eigen(m)
\end{Sinput}
\begin{Soutput}
$values
[1]  3 -1

$vectors
          [,1] [,2]
[1,] 0.4472136    0
[2,] 0.8944272    1
\end{Soutput}
\end{Schunk}
      On remarque que les vecteurs propres obtenus avec \texttt{eigen}
      sont normalisés de sorte que leur norme soit toujours égale à 1.
    \item On a
      \begin{align*}
        \det(\lambda \mat{I} - \mat{A})
        &= \begin{vmatrix}
          \lambda - 10 & 9 \\
          -4 & \lambda + 2
        \end{vmatrix} \\
        &= (\lambda - 10)(\lambda + 2) + 36 \\
        &= (\lambda - 4)^2.
      \end{align*}
      On a donc une seule valeur propre: $\lambda = 4$.  Le système
      d'équations $(4 \mat{I} - \mat{A}) \mat{x} = \mat{0}$ se réduit
      à
      \begin{displaymath}
        \begin{bmatrix} 1 & -\frac{3}{2} \\ 0 & 0 \end{bmatrix}
        \begin{bmatrix} x_1 \\ x_2 \end{bmatrix} =
        \begin{bmatrix} 0 \\ 0 \end{bmatrix},
      \end{displaymath}
      soit $x_1 = 3s/2$ et $x_2 = s$. Une base de vecteurs propres
      correspondant à $\lambda = 4$ est donc $(\frac{3}{2}, 1)$.
      Vérification:
\begin{Schunk}
\begin{Sinput}
> m <- matrix(c(10, 4, -9, -2), nrow = 2)
> eigen(m)
\end{Sinput}
\begin{Soutput}
$values
[1] 4 4

$vectors
           [,1]      [,2]
[1,] -0.8320503 0.8320503
[2,] -0.5547002 0.5547002
\end{Soutput}
\end{Schunk}
    \item On a
      \begin{align*}
        \det(\lambda \mat{I} - \mat{A})
        &= \begin{vmatrix}
          \lambda - 4 & 0 & -1 \\
          2 & \lambda - 1 & 0 \\
          2 & 0 & \lambda - 1
        \end{vmatrix} \\
        &= (\lambda - 3)(\lambda - 1)^2 + 2(\lambda - 1) \\
        &= (\lambda - 1)(\lambda - 2)(\lambda - 3).
      \end{align*}
      Les valeurs propres sont donc $\lambda_1 = 1$, $\lambda_2 = 2$
      et $\lambda_3 = 3$.  En premier lieu, le système d'équations
      $(\mat{I} - \mat{A}) \mat{x} = \mat{0}$ se réduit à
      \begin{displaymath}
        \begin{bmatrix}
          1 & 0 & 0 \\
          0 & 0 & 1 \\
          0 & 0 & 0
        \end{bmatrix}
        \begin{bmatrix} x_1 \\ x_2 \\ x_3 \end{bmatrix} =
        \begin{bmatrix} 0 \\ 0 \\ 0 \end{bmatrix},
      \end{displaymath}
      soit $x_1 = x_3 = 0$ et $x_2 = s$. Une base de vecteurs propres
      correspondant à $\lambda = 1$ est donc $(0, 1, 0)$.
      Deuxièmement, le système d'équations $(2\mat{I} - \mat{A})
      \mat{x} = \mat{0}$ se réduit à
      \begin{displaymath}
        \begin{bmatrix}
          1 &  0 & \frac{1}{2} \\
          0 &  1 & -1 \\
          0 &  0 &  0
        \end{bmatrix}
        \begin{bmatrix} x_1 \\ x_2 \\ x_3 \end{bmatrix} =
        \begin{bmatrix} 0 \\ 0 \\ 0 \end{bmatrix},
      \end{displaymath}
      soit $x_1 = -s/2$, $x_2 = s$ et $x_3 = s$. Une base de vecteurs
      propres correspondant à $\lambda = 2$ est donc $(-\frac{1}{2},
      1, 1)$.  Finalement, le système d'équations $(3\mat{I} -
      \mat{A}) \mat{x} = \mat{0}$ se réduit à
      \begin{displaymath}
        \begin{bmatrix}
          1 &  0 &  1 \\
          0 &  1 & -1 \\
          0 &  0 &  0
        \end{bmatrix}
        \begin{bmatrix} x_1 \\ x_2 \\ x_3 \end{bmatrix} =
        \begin{bmatrix} 0 \\ 0 \\ 0 \end{bmatrix},
      \end{displaymath}
      soit $x_1 = -s$, $x_2 = s$ et $x_3 = s$. Une base de vecteurs
      propres correspondant à $\lambda = 2$ est donc $(-1, 1, 1)$.
      Vérification:
\begin{Schunk}
\begin{Sinput}
> m <- matrix(c(4, -2, -2, 0, 1, 0, 1, 0, 1), nrow = 3)
> eigen(m)
\end{Sinput}
\begin{Soutput}
$values
[1] 3 2 1

$vectors
           [,1]       [,2] [,3]
[1,]  0.5773503 -0.3333333    0
[2,] -0.5773503  0.6666667    1
[3,] -0.5773503  0.6666667    0
\end{Soutput}
\end{Schunk}
    \item On a
      \begin{align*}
        \det(\lambda \mat{I} - \mat{A})
        &= \begin{vmatrix}
          \lambda - 5 &  -6 &  -2 \\
          0 & \lambda + 1 & 8 \\
          -1 &  0 & \lambda + 2
        \end{vmatrix} \\
        &= (\lambda - 5)(\lambda + 1)(\lambda + 2) - 2(\lambda + 1) + 48 \\
        &= \lambda^3 - 2 \lambda^2 - 15 \lambda + 36 \\
        &= (\lambda - 3)^2 (\lambda + 4).
      \end{align*}
      Les valeurs propres sont donc $\lambda_1 = 3$ et $\lambda_2 =
      \lambda_3 = -4$.  En premier lieu, le système d'équations
      $(3\mat{I} - \mat{A}) \mat{x} = \mat{0}$ se réduit à
      \begin{displaymath}
        \begin{bmatrix}
          1 & 0 & -5 \\
          0 & 1 &  2 \\
          0 & 0 &  0
        \end{bmatrix}
        \begin{bmatrix} x_1 \\ x_2 \\ x_3 \end{bmatrix} =
        \begin{bmatrix} 0 \\ 0 \\ 0 \end{bmatrix},
      \end{displaymath}
      soit $x_1 = 5 s$, $x_2 = -2s$ et $x_3 = s$. Une base de vecteurs
      propres correspondant à $\lambda = 3$ est donc $(5, -2, 1)$.
      Deuxièmement, le système d'équations $(-4\mat{I} - \mat{A})
      \mat{x} = \mat{0}$ se réduit à
      \begin{displaymath}
        \begin{bmatrix}
          1 & 0 & 2 \\
          0 & 1 & -\frac{8}{3} \\
          0 & 0 & 0
        \end{bmatrix}
        \begin{bmatrix} x_1 \\ x_2 \\ x_3 \end{bmatrix} =
        \begin{bmatrix} 0 \\ 0 \\ 0 \end{bmatrix},
      \end{displaymath}
      soit $x_1 = -2s$, $x_2 = 8s/3$ et $x_3 = s$. Une base de
      vecteurs propres correspondant à $\lambda = -4$ est donc $(-2,
      \frac{8}{3}, 1)$.  Vérification:
\begin{Schunk}
\begin{Sinput}
> m <- matrix(c(5, 0, 1, 6, -1, 0, 2, -8, -2), nrow = 3)
> eigen(m)
\end{Sinput}
\begin{Soutput}
$values
[1] -4  3  3

$vectors
           [,1]       [,2]       [,3]
[1,]  0.5746958 -0.9128709  0.9128709
[2,] -0.7662610  0.3651484 -0.3651484
[3,] -0.2873479 -0.1825742  0.1825742
\end{Soutput}
\end{Schunk}
    \item On a
      \begin{align*}
        \det(\lambda \mat{I} - \mat{A})
        &= \begin{vmatrix}
          \lambda & 0 &  -2 & 0 \\
          -1 & \lambda &  -1 & 0 \\
          0 & -1 & \lambda + 2 & 0 \\
          0 & 0 &  0 & \lambda - 1
        \end{vmatrix} \\
        &= (\lambda - 1)
        \begin{vmatrix}
          \lambda & 0 &  -2 \\
          -1 & \lambda &  -1 \\
          0 & -1 & \lambda + 2
        \end{vmatrix} \\
        &= (\lambda - 1)(\lambda^3 + 2 \lambda^2 - \lambda - 2) \\
        &= (\lambda - 1)^2 (\lambda + 1)(\lambda + 2).
      \end{align*}
      Les valeurs propres sont donc $\lambda_1 = \lambda_2 = 1$,
      $\lambda_3 = -1$ et $\lambda_4 = -2$.  En premier lieu, le
      système d'équations $(\mat{I} - \mat{A}) \mat{x} = \mat{0}$ se
      réduit à
      \begin{displaymath}
        \begin{bmatrix}
          1 & 0 & -2 & 0 \\
          0 & 1 & -3 & 0 \\
          0 & 0 &  0 & 0 \\
          0 & 0 &  0 & 0 \\
        \end{bmatrix}
        \begin{bmatrix} x_1 \\ x_2 \\ x_3 \\ x_4 \end{bmatrix} =
        \begin{bmatrix} 0 \\ 0 \\ 0 \\ 0 \end{bmatrix},
      \end{displaymath}
      soit $x_1 = 2s$, $x_2 = 3s$, $x_3 = s$ et $x_4 = t$. Or, $(2s,
      3s, s, t) = s(2, 3, 1, 0) + t (0, 0, 0, 1)$. Une base de
      vecteurs propres correspondant à $\lambda = 1$ est donc composée
      des vecteurs $(2, 3, 1, 0)$ et $(0, 0, 0, 1)$.  Deuxièmement, le
      système d'équations $(-\mat{I} - \mat{A}) \mat{x} = \mat{0}$ se
      réduit à
      \begin{displaymath}
        \begin{bmatrix}
          1 & 0 &  2 & 0 \\
          0 & 1 & -1 & 0 \\
          0 & 0 &  0 & 1 \\
          0 & 0 &  0 & 0 \\
        \end{bmatrix}
        \begin{bmatrix} x_1 \\ x_2 \\ x_3 \\ x_4 \end{bmatrix} =
        \begin{bmatrix} 0 \\ 0 \\ 0 \\ 0 \end{bmatrix},
      \end{displaymath}
      soit $x_1 = -2s$, $x_2 = s$, $x_3 = s$ et $x_4 = 0$. Une base de
      vecteurs propres correspondant à $\lambda = -1$ est donc $(-2,
      1, 1, 0)$.  Finalement, le système d'équations $(-2\mat{I} -
      \mat{A}) \mat{x} = \mat{0}$ se réduit à
      \begin{displaymath}
        \begin{bmatrix}
          1 &  0 &  1 &  0 \\
          0 &  1 &  0 &  0 \\
          0 &  0 &  0 &  1 \\
          0 &  0 &  0 &  0 \\
        \end{bmatrix}
        \begin{bmatrix} x_1 \\ x_2 \\ x_3 \\ x_4 \end{bmatrix} =
        \begin{bmatrix} 0 \\ 0 \\ 0 \\ 0 \end{bmatrix},
      \end{displaymath}
      soit $x_1 = -s$, $x_2 = 0$, $x_3 = s$ et $x_4 = 0$. Une base de
      vecteurs propres correspondant à $\lambda = -2$ est donc $(-1,
      0, 1, 0)$.  Vérification:
\begin{Schunk}
\begin{Sinput}
> m <- matrix(c(0, 1, 0, 0, 0, 0, 1, 0, 2, 1, -2, 0, 0, 0, 0, 1), nrow = 4)
> eigen(m)
\end{Sinput}
\begin{Soutput}
$values
[1] -2 -1  1  1

$vectors
              [,1]       [,2] [,3]       [,4]
[1,] -7.071068e-01  0.8164966    0 -0.5345225
[2,]  4.317754e-16 -0.4082483    0 -0.8017837
[3,]  7.071068e-01 -0.4082483    0 -0.2672612
[4,]  0.000000e+00  0.0000000    1  0.0000000
\end{Soutput}
\end{Schunk}
    \end{enumerate}
  \end{sol}
  \begin{rep}
    \begin{enumerate}
    \item $\lambda =  3$ avec base $(\frac{1}{2}, 1)$,
          $\lambda = -1$ avec base $(0, 1)$
    \item $\lambda =  4$ avec base $(\frac{3}{2}, 1)$
    \item $\lambda =  1$ avec base $(0, 1, 0)$,
          $\lambda =  2$ avec base $(-\frac{1}{2}, 1, 1)$,
          $\lambda =  3$ avec base $(-1, 1, 1)$
    \item $\lambda = -4$ avec base $(-2, \frac{8}{3}, 1)$,
          $\lambda =  3$ avec base $(5, -2, 1)$
    \item $\lambda =  1$ avec base $(0, 0, 0, 1)$ et $(2, 3, 1, 0)$,
          $\lambda = -2$ avec base $(-1, 0, 1, 0)$,
          $\lambda = -1$ avec base $(-2, 1, 1, 0)$
    \end{enumerate}
  \end{rep}
\end{exercice}

\begin{exercice}
  \label{ex:valeurspropres:puissance}
  Démontrer que si $\lambda$ est une valeur propre de la matrice
  $\mat{A}$, alors $\lambda^k$ est une valeur propre de $\mat{A}^k$.
  \begin{sol}
    On procède par induction. Premièrement, l'énoncé est clairement
    vrai pour $k = 1$. On suppose par la suite qu'il est vrai pour $k
    = n$, soit que si $\mat{A} \mat{x} = \lambda \mat{x}$, alors
    $\mat{A}^n \mat{x} = \lambda^n \mat{x}$. Ainsi,
    \begin{displaymath}
      \mat{A}^{n + 1} \mat{x} =
      \mat{A}(\mat{A}^n \mat{x}) =
      \mat{A}(\lambda^n \mat{x}) =
      \lambda^n (\mat{A} \mat{x}) =
      \lambda^n (\lambda \mat{x}) =
      \lambda^{n + 1} \mat{x},
    \end{displaymath}
    d'où l'énoncé est vrai pour $k = n + 1$. Ceci complète la preuve.
  \end{sol}
\end{exercice}

\begin{exercice}
  Trouver les valeurs et vecteurs propres de $\mat{A}^{25}$ si
  \begin{displaymath}
    \mat{A} =
    \begin{bmatrix}
      -1 & -2 & -2 \\
       1 &  2 &  1 \\
      -1 & -1 &  0
    \end{bmatrix}.
  \end{displaymath}
  \begin{sol}
    On va utiliser les résultats de l'exercice
    \ref{chap:valeurspropres}.\ref{ex:valeurspropres:puissance}. Tout d'abord,
    \begin{align*}
      \det(\lambda \mat{I} - \mat{A})
      &= \begin{vmatrix}
        \lambda + 1 & 2 & 2 \\
        -1 &  \lambda - 2 &  1 \\
        1 & 1 & \lambda
      \end{vmatrix} \\
      &= (\lambda + 1)(\lambda - 1)^2,
    \end{align*}
    d'où les valeurs propres de $\mat{A}$ sont $\lambda_1 = \lambda_2
    = 1$ et $\lambda_3 = -1$. Par conséquent, les valeurs propres de
    $\mat{A}^{25}$ sont $\lambda_1 = \lambda_2 = 1^{25} = 1$ et
    $\lambda_3 = (-1)^{25} = -1$.  Toujours par le résultat de
    l'exercice \ref{chap:valeurspropres}.\ref{ex:valeurspropres:puissance}, les vecteurs propres
    de $\mat{A}$ correspondant à $\lambda = 1$ et $\lambda = -1$ sont
    également des vecteurs propres de $\mat{A}^{25}$. Or, le système
    d'équations $(\mat{I} - \mat{A}) \mat{x} = \mat{0}$ se réduit à
      \begin{displaymath}
        \begin{bmatrix}
          1 &  1 & 1 \\
          0 &  0 & 0 \\
          0 &  0 & 0
        \end{bmatrix}
        \begin{bmatrix} x_1 \\ x_2 \\ x_3 \end{bmatrix} =
        \begin{bmatrix} 0 \\ 0 \\ 0 \end{bmatrix},
      \end{displaymath}
      soit $x_1 = -s - t$, $x_2 = s$ et $x_3 = t$. Puisque $(-s - t,
      s, t) = s(-1, 1, 0) + t(-1, 0, 1)$, une base de vecteurs propres
      correspondant à $\lambda = 1$ est composée de $(-1, 1, 0)$ et
      $(-1, 0, 1)$.  D'autre part, le système d'équations $(-\mat{I} -
      \mat{A}) \mat{x} = \mat{0}$ se réduit à
      \begin{displaymath}
        \begin{bmatrix}
          1 & 0 & -2 \\
          0 & 1 &  1 \\
          0 & 0 &  0
        \end{bmatrix}
        \begin{bmatrix} x_1 \\ x_2 \\ x_3 \end{bmatrix} =
        \begin{bmatrix} 0 \\ 0 \\ 0 \end{bmatrix},
      \end{displaymath}
      soit $x_1 = 2s$, $x_2 = -s$ et $x_3 = s$. Une base de vecteurs
      propres correspondant à $\lambda = -1$ est donc $(2, -1, 1)$.
  \end{sol}
  \begin{rep}
    $\lambda =  1$ avec base $(-1, 1, 0)$ et $(-1, 0, 1)$,
    $\lambda = -1$ avec base $(2, -1, 1)$
  \end{rep}
\end{exercice}

\begin{exercice}
  Soit
  \begin{displaymath}
    (x - x_1)(x - x_2) \cdots (x - x_n) = x^n + c_1 x^{n - 1} + \dots + c_n.
  \end{displaymath}
  Démontrer par induction les identités suivantes.
  \begin{enumerate}
  \item $c_1 = - \sum_{i=1}^n x_i$
  \item $c_n = (-1)^n \prod_{i=1}^n x_i$
  \end{enumerate}
  \begin{sol}
    On peut démontrer les deux résultats simultanément.  Tout d'abord,
    les résultats sont clairement vrais pour $n = 1$, c'est-à-dire
    $c_1 = c_n = - x_1$. On suppose ensuite que les résultats sont
    vrais pour $n = k$, soit
    \begin{displaymath}
      \prod_{i = 1}^k (x - x_i) =
      x^k + \left(- \sum_{i = 1}^k x_i \right) x^{k - 1} + \dots +
      \prod_{i = 1}^k x_i.
    \end{displaymath}
    Si $n = k + 1$, on a
    \begin{align*}
      \prod_{i = 1}^{k + 1} (x - x_i)
      &=
      \left(
        \prod_{i = 1}^k x - x_i
      \right)
      (x - x_{k + 1}) \\
      &=
      \left(
        x^k + \left(- \sum_{i = 1}^k x_i \right) x^{k - 1} + \dots +
        \prod_{i = 1}^k x_i
      \right)
      (x - x_{k + 1}) \\
      &= x^{k + 1} + \left( - \sum_{i = 1}^k x_i + x_{k + 1} \right)
      x^k + \dots + x_{k + 1} \prod_{i = 1}^k x_i \\
      &= x^{k + 1} + \left( - \sum_{i = 1}^{k + 1} x_i \right) x^k +
      \dots + \prod_{i = 1}^{k + 1} x_i.
    \end{align*}
    Les résultats sont donc vrais pour $n = k + 1$. Ceci complète la
    preuve.
  \end{sol}
\end{exercice}

\begin{exercice}
  Démontrer que l'équation caractéristique d'une matrice
  $\mat{A}_{2 \times 2}$ est
  \begin{displaymath}
    \lambda^2 - \tr(\mat{A}) \lambda + \det(\mat{A}) = 0.
  \end{displaymath}
  \begin{sol}
    Soit
    \begin{displaymath}
      \mat{A} =
      \begin{bmatrix}
        a_{11} & a_{12} \\ a_{21} & a_{22}
      \end{bmatrix}.
    \end{displaymath}
    Le polynôme caractéristique de cette matrice est
    \begin{align*}
      \det(\lambda \mat{I} - \mat{A})
      &=
      \begin{vmatrix}
        \lambda - a_{11} & -a_{12} \\ -a_{21} & \lambda - a_{22}
      \end{vmatrix} \\
      &= (\lambda - a_{11})(\lambda - a_{22}) - a_{12} a_{21} \\
      &= \lambda^2 - (a_{11} + a_{22}) \lambda +
      (a_{11} a_{22} - a_{12} a_{21}) \\
      &= \lambda^2 - \tr(\mat{A}) \lambda + \det(\mat{A}),
    \end{align*}
    d'où l'équation caractéristique est $\lambda^2 - \tr(\mat{A})
    \lambda + \det(\mat{A}) = 0$.
  \end{sol}
\end{exercice}

\begin{exercice}
  \label{ex:valeurspropres:inverse}
  Démontrer que si $\lambda$ est une valeur propre de la matrice
  inversible $\mat{A}$ et que $\mat{x}$ est un vecteur propre
  correspondant, alors $\lambda^{-1}$ est une valeur propre de
  $\mat{A}^{-1}$ et $\mat{x}$ est un vecteur propre correspondant.
  \begin{sol}
    On a $\mat{A} \mat{x} = \lambda \mat{x}$. En multipliant de part
    et d'autre (par la gauche) par $\mat{A}^{-1}$, on obtient $\mat{x}
    = \lambda \mat{A}^{-1} \mat{x}$, soit $\mat{A}^{-1} \mat{x} =
    \lambda^{-1} \mat{x}$. Par conséquent, $\lambda^{-1}$ est une
    valeur propre de $\mat{A}^{-1}$ et $\mat{x}$ est un vecteur propre
    correspondant.
  \end{sol}
\end{exercice}

\begin{exercice}
  Trouver les valeurs propres et les bases de vecteurs propres de
  $\mat{A}^{-1}$, où
  \begin{displaymath}
    \mat{A} =
    \begin{bmatrix}
      -2 & 2 & 3 \\
      -2 & 3 & 2 \\
      -4 & 2 & 5
    \end{bmatrix}
  \end{displaymath}
  \begin{sol}
    On utilise le résultat de l'exercice
    \ref{chap:valeurspropres}.\ref{ex:valeurspropres:inverse} pour
    éviter de devoir calculer l'inverse de la matrice. Le polynôme
    caractéristique de la matrice $\mat{A}$ est
    \begin{align*}
      \det(\lambda \mat{I} - \mat{A})
      &= \begin{vmatrix}
        \lambda + 2 & -2 & -3 \\
        2 &  \lambda - 3 & -2 \\
        4 & -2 & \lambda - 5
      \end{vmatrix} \\
      &= \lambda^3 - 6 \lambda^2 + 11 \lambda - 6 \\
      &= (\lambda - 1)(\lambda - 2)(\lambda - 3),
    \end{align*}
    d'où les valeurs propres de $\mat{A}$ sont $\lambda_1 = 1$,
    $\lambda_2 = 2$ et $\lambda_3 = -1$. Par conséquent, les valeurs
    propres de $\mat{A}^{-1}$ sont $\lambda_1 = 1$, $\lambda_2 =
    \frac{1}{2}$ et $\lambda_3 = \frac{1}{3}$. Toujours par le
    résultat de l'exercice
    \ref{chap:valeurspropres}.\ref{ex:valeurspropres:inverse}, les
    vecteurs propres de $\mat{A}$ correspondant à $\lambda = 1$,
    $\lambda = 2$ et $\lambda = 3$ sont également des vecteurs propres
    de $\mat{A}^{-1}$ correspondant à $\lambda_1 = 1$, $\lambda_2 =
    \frac{1}{2}$ et $\lambda_3 = \frac{1}{3}$. Or, le système
    d'équations $(\mat{I} - \mat{A}) \mat{x} = \mat{0}$ se réduit à
    \begin{displaymath}
      \begin{bmatrix}
        1 & 0 & -1 \\
        0 & 1 &  0 \\
        0 & 0 &  0
      \end{bmatrix}
      \begin{bmatrix} x_1 \\ x_2 \\ x_3 \end{bmatrix} =
      \begin{bmatrix} 0 \\ 0 \\ 0 \end{bmatrix},
    \end{displaymath}
    soit $x_1 = s$, $x_2 = 0$ et $x_3 = s$. Une base de vecteurs
    propres correspondant à $\lambda = 1$ est donc $(1, 0, 1)$.
    D'autre part, le système d'équations $(2\mat{I} - \mat{A}) \mat{x}
    = \mat{0}$ se réduit à
    \begin{displaymath}
      \begin{bmatrix}
        1 & -\frac{1}{2} & 0 \\
        0 &  0 & 1 \\
        0 &  0 & 0
      \end{bmatrix}
      \begin{bmatrix} x_1 \\ x_2 \\ x_3 \end{bmatrix} =
      \begin{bmatrix} 0 \\ 0 \\ 0 \end{bmatrix},
    \end{displaymath}
    soit $x_1 = s/2$, $x_2 = s$ et $x_3 = 0$. Une base de vecteurs
    propres correspondant à $\lambda = 2$ (ou $\lambda = \frac{1}{2}$)
    est donc $(\frac{1}{2}, 1, 0)$.  Finalement, le système
    d'équations $(3\mat{I} - \mat{A}) \mat{x} = \mat{0}$ se réduit à
    \begin{displaymath}
      \begin{bmatrix}
        1 &  0 & -1 \\
        0 &  1 & -1 \\
        0 &  0 &  0
      \end{bmatrix}
      \begin{bmatrix} x_1 \\ x_2 \\ x_3 \end{bmatrix} =
      \begin{bmatrix} 0 \\ 0 \\ 0 \end{bmatrix},
    \end{displaymath}
    soit $x_1 = s$, $x_2 = s$ et $x_3 = s$. Une base de vecteurs
    propres correspondant à $\lambda = 3$ (ou $\lambda = \frac{1}{3}$)
    est donc $(1, 1, 1)$.
  \end{sol}
  \begin{rep}
    $\lambda = 1$ avec base $(1, 0, 1)$,
    $\lambda = \frac{1}{2}$ avec base $(\frac{1}{2}, 1, 0)$,
    $\lambda = \frac{1}{3}$ avec base $(1, 1, 1)$
  \end{rep}
\end{exercice}

\begin{exercice}
  Démontrer que tout vecteur est un vecteur propre de la matrice
  identité correspondant à la valeur propre $\lambda = 1$.
  \begin{sol}
    Le résultat découle simplement du fait que l'équation $\mat{I}
    \mat{x} = \mat{x}$ est vraie pour tout vecteur $\mat{x}$.
  \end{sol}
\end{exercice}

\begin{exercice}
  Pour chacune des matrices $\mat{A}$ ci-dessous:
  \begin{enumerate}[i)]
  \item trouver les valeurs propres de la matrice;
  \item trouver le rang de la matrice $\lambda \mat{I} - \mat{A}$ pour
    chaque valeur propre $\lambda$;
  \item déterminer si la matrice est diagonalisable;
  \item si la matrice est diagonalisable, trouver la matrice $\mat{P}$
    qui diagonalise $\mat{A}$ et $\mat{P}^{-1} \mat{AP}$;
  \item vérifier les réponses en iv) avec la fonction \texttt{eigen}
    de R.
  \end{enumerate}
  \begin{enumerate}
    \begin{multicols}{2}
    \item $\begin{bmatrix} 2 & 0 \\ 1 & 2 \end{bmatrix}$
    \item $\begin{bmatrix}
        4 &  0 &  1 \\
        2 &  3 &  2 \\
        1 & 0 & 4
      \end{bmatrix}$
    \end{multicols}
    \begin{multicols}{2}
    \item $\begin{bmatrix}
        3 &  0 &  0 \\
        0 &  2 &  0 \\
        0 & 1 & 2
      \end{bmatrix}$
    \item $\begin{bmatrix}
        -1 &  4 & -2 \\
        -3 &  4 &  0 \\
        -3 & 1 & 3
      \end{bmatrix}$
    \end{multicols}
    \begin{multicols}{2}
    \item $\begin{bmatrix}
        -2 &  0 &  0 &  0 \\
        0 & -2 &  5 & -5 \\
        0 &  0 &  3 &  0 \\
        0 & 0 & 0 & 3
      \end{bmatrix}$
    \end{multicols}
  \end{enumerate}
  \begin{sol}
    \begin{enumerate}
    \item Le polynôme caractéristique est $\lambda^2 - 4 \lambda + 4 =
      (\lambda - 2)^2$, donc les valeurs propres sont $\lambda_1 =
      \lambda_2 = 2$. On a donc
      \begin{displaymath}
        2 \mat{I} - \mat{A} =
        \begin{bmatrix}
          0 & 0 \\ -1 & 0
        \end{bmatrix},
      \end{displaymath}
      d'où $\mathrm{rang}(2 \mat{I} - \mat{A}) = 1$. Puisque la
      matrice $\mat{A}$ ne possède qu'un seul vecteur propre, elle
      n'est pas diagonalisable.  Vérification:
\begin{Schunk}
\begin{Sinput}
> m <- matrix(c(2, 1, 0, 2), nrow = 2)
> eigen(m)
\end{Sinput}
\begin{Soutput}
$values
[1] 2 2

$vectors
     [,1]          [,2]
[1,]    0  4.440892e-16
[2,]    1 -1.000000e+00
\end{Soutput}
\end{Schunk}
    \item Le polynôme caractéristique est $(\lambda - 3)(\lambda^2 - 8
      \lambda + 15) = (\lambda - 3)^2 (\lambda - 5)$, donc les valeurs
      propres sont $\lambda_1 = \lambda_2 = 3$ et $\lambda_3 = 5$.  La
      forme échelonnée de la matrice $3 \mat{I} - \mat{A}$ est
      \begin{displaymath}
        \begin{bmatrix}
          1 & 0 & 1 \\
          0 & 0 & 0 \\
          0 & 0 & 0
        \end{bmatrix},
      \end{displaymath}
      d'où $\mathrm{rang}(3 \mat{I} - \mat{A}) = 1$. La base de
      vecteurs propres correspondant à $\lambda = 3$ est composée des
      vecteurs $(-1, 0, 1)$ et $(0, 1, 0)$.  D'autre part, la forme
      échelonnée de la matrice $5 \mat{I} - \mat{A}$ est
      \begin{displaymath}
        \begin{bmatrix}
          1 &  0 & -1 \\
          0 &  1 & -2 \\
          0 &  0 &  0
        \end{bmatrix}
      \end{displaymath}
      d'où $\mathrm{rang}(5 \mat{I} - \mat{A}) = 2$. La base de
      vecteurs propres correspondant à $\lambda = 5$ est $(1, 2, 1)$.
      Bien que les valeurs propres de la matrice $\mat{A}$ ne sont pas
      toutes distinctes, les vecteurs propres $(1, 0, -1)$, $(0, 1,
      0)$ et $(1, 2, 1)$ sont linéairement indépendants. Par
      conséquent, la matrice
      \begin{align*}
        \mat{P}
        &=
        \begin{bmatrix}
          -1 & 0 & 1 \\ 0 & 1 & 2 \\ 1 & 0 & 1
        \end{bmatrix} \\
        \intertext{diagonalise $\mat{A}$ et}
        \mat{P}^{-1} \mat{AP}
        &=
        \begin{bmatrix}
          3 & 0 & 0 \\ 0 & 3 & 0 \\ 0 & 0 & 5
        \end{bmatrix}.
      \end{align*}
      Vérification:
\begin{Schunk}
\begin{Sinput}
> m <- matrix(c(4, 2, 1, 0, 3, 0, 1, 2, 4), nrow = 3)
> eigen(m)
\end{Sinput}
\begin{Soutput}
$values
[1] 5 3 3

$vectors
          [,1] [,2]       [,3]
[1,] 0.4082483    0 -0.7071068
[2,] 0.8164966    1  0.0000000
[3,] 0.4082483    0  0.7071068
\end{Soutput}
\end{Schunk}
    \item La matrice étant triangulaire, on sait immédiatement que les
      valeurs propres sont $\lambda_1 = 3$ et $\lambda_2 = \lambda_3 =
      2$. La forme échelonnée de la matrice $3 \mat{I} - \mat{A}$ est
      \begin{displaymath}
        \begin{bmatrix}
          0 & 1 & 0 \\
          0 & 0 & 1 \\
          0 & 0 & 0
        \end{bmatrix},
      \end{displaymath}
      d'où $\mathrm{rang}(3 \mat{I} - \mat{A}) = 2$. La base de
      vecteurs propres correspondant à $\lambda = 3$ est $(1, 0, 0)$.
      D'autre part, la forme échelonnée de la matrice $2 \mat{I} -
      \mat{A}$ est
      \begin{displaymath}
        \begin{bmatrix}
          1 &  0 &  0 \\
          0 &  1 &  0 \\
          0 &  0 &  0
        \end{bmatrix}
      \end{displaymath}
      d'où $\mathrm{rang}(2 \mat{I} - \mat{A}) = 2$. La base de
      vecteurs propres correspondant à $\lambda = 2$ est $(0, 0, 1)$.
      Par conséquent, la matrice $\mat{A}$ n'a pas trois vecteurs
      linéairement indépendants, donc elle n'est pas diagonalisable.
      Vérification:
\begin{Schunk}
\begin{Sinput}
> m <- matrix(c(3, 0, 0, 0, 2, 1, 0, 0, 2), nrow = 3)
> eigen(m)
\end{Sinput}
\begin{Soutput}
$values
[1] 3 2 2

$vectors
     [,1] [,2]          [,3]
[1,]    1    0  0.000000e+00
[2,]    0    0  4.440892e-16
[3,]    0    1 -1.000000e+00
\end{Soutput}
\end{Schunk}
    \item Le polynôme caractéristique est $\lambda^3 - 6 \lambda^2 +
      11 \lambda - 6 = (\lambda - 1)(\lambda - 2)(\lambda - 3)$, donc
      les valeurs propres sont $\lambda_1 = 1$, $\lambda_2 = 2$ et
      $\lambda_3 = 3$. Les valeurs propres étant distinctes, la
      matrice est diagonalisable. Or, la forme échelonnée de la
      matrice $\mat{I} - \mat{A}$ est
      \begin{displaymath}
        \begin{bmatrix}
          1 & 0 & -1 \\
          0 & 1 & -1 \\
          0 & 0 &  0
        \end{bmatrix},
      \end{displaymath}
      d'où $\mathrm{rang}(\mat{I} - \mat{A}) = 2$. La base de vecteurs
      propres correspondant à $\lambda = 1$ est $(1, 1, 1)$.
      Deuxièmement, la forme échelonnée de la matrice $2 \mat{I} -
      \mat{A}$ est
      \begin{displaymath}
        \begin{bmatrix}
          1 &  0 & -\frac{2}{3} \\
          0 &  1 &  -1 \\
          0 &  0 &  0
        \end{bmatrix}
      \end{displaymath}
      d'où $\mathrm{rang}(2 \mat{I} - \mat{A}) = 2$ et la base de
      vecteurs propres correspondant à $\lambda = 2$ est $(2, 3, 3)$.
      Enfin, la forme échelonnée de la matrice $3 \mat{I} - \mat{A}$
      est
      \begin{displaymath}
        \begin{bmatrix}
          1 &  0 & -\frac{1}{4} \\
          0 &  1 & -\frac{3}{4} \\
          0 &  0 & 0
        \end{bmatrix}
      \end{displaymath}
      d'où $\mathrm{rang}(3 \mat{I} - \mat{A}) = 2$ et la base de
      vecteurs propres correspondant à $\lambda = 3$ est $(1, 3, 4)$.
      Par conséquent,
      \begin{align*}
        \mat{P}
        &=
        \begin{bmatrix}
          1 & 2 & 1 \\ 1 & 3 & 3 \\ 1 & 3 & 4
        \end{bmatrix} \\
        \intertext{diagonalise $\mat{A}$ et}
        \mat{P}^{-1} \mat{AP}
        &=
        \begin{bmatrix}
          1 & 0 & 0 \\ 0 & 2 & 0 \\ 0 & 3 & 0
        \end{bmatrix}.
      \end{align*}
      Vérification:
\begin{Schunk}
\begin{Sinput}
> m <- matrix(c(-1, -3, -3, 4, 4, 1, -2, 0, 3), nrow = 3)
> eigen(m)
\end{Sinput}
\begin{Soutput}
$values
[1] 3 2 1

$vectors
          [,1]      [,2]       [,3]
[1,] 0.1961161 0.4264014 -0.5773503
[2,] 0.5883484 0.6396021 -0.5773503
[3,] 0.7844645 0.6396021 -0.5773503
\end{Soutput}
\end{Schunk}
    \item La matrice est triangulaire: les valeurs propres sont donc
      $\lambda_1 = \lambda_2 = -2$ et $\lambda_3 = \lambda_4 = 3$. La
      forme échelonnée de la matrice $-2 \mat{I} - \mat{A}$ est
      \begin{displaymath}
        \begin{bmatrix}
          0 & 0 & 1 & 0 \\
          0 & 0 & 0 & 1 \\
          0 & 0 & 0 & 0 \\
          0 & 0 & 0 & 0 \\
        \end{bmatrix},
      \end{displaymath}
      d'où $\mathrm{rang}(-2 \mat{I} - \mat{A}) = 2$ et la base de
      vecteurs propres correspondant à $\lambda = -2$ est composée des
      vecteurs $(1, 0, 0, 0)$ et $(0, 1, 0, 0)$.  D'autre part, la
      forme échelonnée de la matrice $3 \mat{I} - \mat{A}$ est
      \begin{displaymath}
        \begin{bmatrix}
          1 & 0 &  0 & 0 \\
          0 & 1 & -1 & 1 \\
          0 & 0 &  0 & 0 \\
          0 & 0 &  0 & 0 \\
        \end{bmatrix}
      \end{displaymath}
      d'où $\mathrm{rang}(2 \mat{I} - \mat{A}) = 2$ et la base de
      vecteurs propres correspondant à $\lambda = 3$ est composée des
      vecteurs $(0, 1, 1, 0)$ et $(0, -1, 0, 1)$.  Les valeurs propres
      de la matrice $\mat{A}$ ne sont pas toutes distinctes, mais les
      vecteurs propres $(1, 0, 0, 0)$, $(0, 1, 0, 0)$, $(0, 1, 1, 0)$
      et $(0, -1, 0, 1)$ sont linéairement indépendants. Par
      conséquent, la matrice
      \begin{align*}
        \mat{P}
        &=
        \begin{bmatrix}
          1 & 0 & 0 &  0 \\
          0 & 1 & 1 & -1 \\
          0 & 0 & 1 &  0 \\
          0 & 0 & 0 &  1 \\
        \end{bmatrix} \\
        \intertext{diagonalise $\mat{A}$ et}
        \mat{P}^{-1} \mat{AP}
        &=
        \begin{bmatrix}
          -2 &  0 & 0 & 0 \\
           0 & -2 & 0 & 0 \\
           0 &  0 & 3 & 0 \\
           0 &  0 & 0 & 3 \\
        \end{bmatrix}.
      \end{align*}
      Vérification:
\begin{Schunk}
\begin{Sinput}
> m <- matrix(c(-2, 0, 0, 0, 0, -2, 0, 0, 0, 5, 3, 0, 0, -5, 0, 3), nrow = 4)
> eigen(m)
\end{Sinput}
\begin{Soutput}
$values
[1]  3  3 -2 -2

$vectors
          [,1]       [,2] [,3] [,4]
[1,] 0.0000000  0.0000000    1    0
[2,] 0.7071068 -0.7071068    0    1
[3,] 0.7071068  0.0000000    0    0
[4,] 0.0000000  0.7071068    0    0
\end{Soutput}
\end{Schunk}
    \end{enumerate}
  \end{sol}
  \begin{rep}
    \begin{enumerate}
    \item pas diagonalisable
    \item $\mat{P} =
      \begin{bmatrix}
         1 & 0 & 1 \\
         0 & 1 & 2 \\
        -1 & 0 & 1
      \end{bmatrix}$,
      $\mat{P}^{-1} \mat{AP} =
      \begin{bmatrix}
        3 & 0 & 0 \\
        0 & 3 & 0 \\
        0 & 0 & 5
      \end{bmatrix}$
    \item pas diagonalisable
    \item $\mat{P} =
      \begin{bmatrix}
        1 & 2 & 1 \\
        1 & 3 & 3 \\
        1 & 3 & 4
      \end{bmatrix}$,
      $\mat{P}^{-1} \mat{AP} =
      \begin{bmatrix}
        1 & 0 & 0 \\
        0 & 2 & 0 \\
        0 & 0 & 3
      \end{bmatrix}$
    \item $\mat{P} =
      \begin{bmatrix}
          1 & 0 & 0 &  0 \\
          0 & 1 & 1 & -1 \\
          0 & 0 & 1 &  0 \\
          0 & 0 & 0 &  1 \\
      \end{bmatrix}$,
      $\mat{P}^{-1} \mat{AP} =
      \begin{bmatrix}
         -2 &  0 & 0 & 0 \\
          0 & -2 & 0 & 0 \\
          0 &  0 & 3 & 0 \\
          0 &  0 & 0 & 3
       \end{bmatrix}$
    \end{enumerate}
  \end{rep}
\end{exercice}

\Closesolutionfile{reponses}
\Closesolutionfile{solutions}

%%%
%%% Insérer les réponses
%%%
\input{reponses-valeurs_propres}


%%% Local Variables:
%%% mode: latex
%%% TeX-master: methodes_numeriques-partie_4
%%% End:
