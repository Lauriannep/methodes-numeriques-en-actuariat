\chapter{Méthodes de résolution de systèmes d'équations linéaires}
\label{chap:decomposition}

\begin{objectifs}
\item Comparer la rapidité des méthodes usuelles de résolutions des
  systèmes d'équations linéaires.
\item Calculer la solution d'un système d'équations linéaires par la
  décomposition $LU$.
\end{objectifs}

L'utilisation des matrices de loin la plus fréquente en actuariat
consiste à résoudre des systèmes d'équations linéaires du type
\begin{displaymath}
  \mat{A} \mat{x} = \mat{b}.
\end{displaymath}
Nous avons jusqu'à maintenant étudié diverses façons --- toutes
équivalentes d'un point de vue mathématique --- d'obtenir la solution
\begin{displaymath}
  \mat{x} = \mat{A}^{-1} \mat{b}.
\end{displaymath}
On peut maintenant se demander laquelle est la plus efficace d'un
point de vue informatique, surtout lorsque le système compte un grand
nombre d'équations.


\section{Comparaison du nombre d'opérations}
\label{sec:decomposition:nombre_operations}

Les principales méthodes de résolution d'un système d'équations
linéaires sont:
\begin{enumerate}
\item l'élimination gaussienne avec substitution successive;
\item l'élimination de Gauss--Jordan;
\item le calcul de $\mat{A}^{-1}$, puis de $\mat{x} = \mat{A}^{-1}
  \mat{b}$.
\end{enumerate}
Le calcul de $\mat{A}^{-1}$ peut quant à lui s'effectuer par
transformation de $[\mat{A}|\mat{I}]$ en $[\mat{I}|\mat{A}^{-1}]$, ou
par la méthode des cofacteurs.

L'élément décisif dans la comparaison des temps de calcul de ces
diverses méthodes de calcul est le nombre d'opérations arithmétiques
requis. Le tableau~\ref{tab:decomposition:nb_oper} présente l'ordre de
grandeur (et non le nombre exact) du nombre de multiplications et de
divisions nécessaires pour obtenir une réponse avec chacune des
méthodes ci-dessus. On se concentre sur les multiplications et
divisions, sachant que ces opérations coûtent plus cher en temps de
calcul que les additions et les soustractions.
\begin{table}
  \centering
  \begin{tabular}{lc}
    \toprule
    Méthode & Nombre d'opérations \\
    \midrule
    Élimination gaussienne & $n^3/3$ \\
    Élimination de Gauss--Jordan & $n^3/3$ \\
    Transformation de $[\mat{A}|\mat{I}]$ en $[\mat{I}|\mat{A}^{-1}]$ &
    $n^3$ \\
    Calcul de $\mat{x} = \mat{A}^{-1} \mat{b}$ & $n^3$ \\
    \bottomrule
  \end{tabular}
  \caption{Nombre approximatif de multiplications et divisions pour
    résoudre le système d'équations à $n$ équations et $n$ inconnues
    $\mat{A} \mat{x} = \mat{b}$}
  \label{tab:decomposition:nb_oper}
\end{table}

On constate au tableau~\ref{tab:decomposition:nb_oper} que
l'élimination gaussienne avec substitution successive et l'élimination
de Gauss--Jordan sont les méthodes les plus rapides, leur avantage
augmentant rapidement avec la taille du système d'équations.


\section{Décomposition LU}
\label{sec:decomposition:decomposition}

Le principal inconvénient des méthodes d'élimination réside dans le
fait qu'il faut connaître le vecteur des coefficients $\mat{b}$ au
moment d'effectuer les calculs. Si l'on souhaite résoudre le système
$\mat{Ax} = \mat{b}$ pour un nouveau vecteur $\mat{b}$, il faut
répéter la procédure depuis le début.

Les principales routines de résolution de systèmes d'équations
linéaires disponibles dans les divers outils informatiques (R, Maple,
Matlab, Mathematica, etc.) reposent donc plutôt sur la technique de la
décomposition $LU$, une variante de l'élimination gaussienne ne
nécessitant pas de connaître d'avance le vecteur $\mat{b}$.

L'idée est très simple: si la matrice $\mat{A}$ peut être factorisée
en un produit de matrices $n \times n$
\begin{displaymath}
  \mat{A} = \mat{L} \mat{U},
\end{displaymath}
où $\mat{L}$ est une matrice triangulaire inférieure et $\mat{U}$ une
matrice triangulaire supérieure, alors on obtient le système
d'équations linéaires
\begin{equation}
  \mat{L} \mat{U} \mat{x} = \mat{b}.
  \label{eq:decomposition:LUx}
\end{equation}
Or, en posant
\begin{equation}
  \mat{U} \mat{x} = \mat{y}
  \label{eq:decomposition:Ux}
\end{equation}
on peut réécrire \eqref{eq:decomposition:LUx} comme
\begin{equation}
  \mat{L} \mat{y} = \mat{b}.
\end{equation}
La solution $\mat{y}$ de ce dernier système d'équations est simple à
obtenir par substitutions successives. De même, une fois $\mat{y}$
connu, le vecteur $\mat{x}$ est obtenu en résolvant
\eqref{eq:decomposition:Ux}, toujours par simples substitutions.

\begin{exemple}
  Soit le système d'équations linéaires
  \begin{displaymath}
    \begin{bmatrix}
       2 &  6 & 2 \\
      -3 & -8 & 0 \\
       4 &  9 & 2
    \end{bmatrix}
    \begin{bmatrix}
      x_1 \\ x_2 \\ x_3
    \end{bmatrix}
    =
    \begin{bmatrix}
      2 \\ 2 \\ 3
    \end{bmatrix}.
  \end{displaymath}
  On peut démontrer que
  \begin{displaymath}
    \mat{A} =
    \begin{bmatrix}
       2 &  6 & 2 \\
      -3 & -8 & 0 \\
       4 &  9 & 2
    \end{bmatrix}
    =
    \begin{bmatrix}
       2 &  0 & 0 \\
      -3 &  1 & 0 \\
       4 & -3 & 7
    \end{bmatrix}
    \begin{bmatrix}
      1 & 3 & 1 \\
      0 & 1 & 3 \\
      0 & 0 & 1
    \end{bmatrix},
  \end{displaymath}
  soit $\mat{A} = \mat{L} \mat{U}$ avec
  \begin{align*}
    \mat{L} &=
    \begin{bmatrix}
       2 &  0 & 0 \\
      -3 &  1 & 0 \\
       4 & -3 & 7
    \end{bmatrix} \\
    \intertext{et}
    \mat{U} &=
    \begin{bmatrix}
      1 & 3 & 1 \\
      0 & 1 & 3 \\
      0 & 0 & 1
    \end{bmatrix}.
  \end{align*}
  On a donc $\mat{A} \mat{x} = \mat{L} \mat{U} \mat{x} =
  \mat{b}$. En posant $\mat{U} \mat{x} = \mat{y}$, on réécrit le
  système d'équations sous la forme $\mat{L} \mat{y} = \mat{b}$, soit
  \begin{displaymath}
    \begin{bmatrix}
       2 &  0 & 0 \\
      -3 &  1 & 0 \\
       4 & -3 & 7
    \end{bmatrix}
    \begin{bmatrix}
      y_1 \\ y_2 \\ y_3
    \end{bmatrix}
    =
    \begin{bmatrix}
      2 \\ 2 \\ 3
    \end{bmatrix}.
  \end{displaymath}
  Par substitution successive, on trouve
  \begin{align*}
    y_1 &= 1 \\
    y_2 &= 2 + 3 y_1 = 5 \\
    y_3 &= \frac{3 - 4 y_1 + 3 y_2}{7} = 2.
  \end{align*}
  Pour trouver la solution du système d'équations original, il suffit
  maintenant de résoudre $\mat{U} \mat{x} = \mat{y}$, soit
  \begin{displaymath}
    \begin{bmatrix}
      1 & 3 & 1 \\
      0 & 1 & 3 \\
      0 & 0 & 1
    \end{bmatrix}
    \begin{bmatrix}
      x_1 \\ x_2 \\ x_3
    \end{bmatrix}
    =
    \begin{bmatrix}
      1 \\ 5 \\ 2
    \end{bmatrix}.
  \end{displaymath}
  On obtient alors
  \begin{align*}
    x_3 &= 2 \\
    x_2 &= 5 - 3 x_3 = -1 \\
    x_1 &= 1 - 3 x_2 - x_3 = 2.
  \end{align*}
  \qed
\end{exemple}

L'essentiel des calculs dans la décomposition $LU$ se trouve dans la
factorisation de la matrice $\mat{A}$ en un produit de matrices
triangulaires. Pour justifier la technique, supposons que l'on réduit
la matrice $\mat{A}$ sous forme échelonnée par une série d'opérations
élémentaires sur les lignes. On peut donc trouver des matrices
élémentaires $\mat{E}_1, \dots, \mat{E}_k$ tel que
\begin{displaymath}
  \mat{E}_k \cdots \mat{E}_1 \mat{A} = \mat{U}.
\end{displaymath}
Par le théorème \ref{thm:revision:matrices_elementaires}, l'inverse
d'une matrice élémentaire existe et est également une matrice
élémentaire, d'où
\begin{displaymath}
  \mat{A} = \mat{E}_1^{-1} \cdots \mat{E}_k^{-1} \mat{U}
\end{displaymath}
et donc
\begin{displaymath}
  \mat{L} = \mat{E}_1^{-1} \cdots \mat{E}_k^{-1}.
\end{displaymath}
Cette dernière matrice est triangulaire inférieure \emph{à condition
  de ne pas échanger des lignes} lors de la réduction de $\mat{A}$
vers $\mat{U}$.  Il existe un algorithme simple pour construire la
matrice $\mat{L}$ sans devoir effectuer le produit des matrices
élémentaires inverses; consulter \citet[section 9.9]{Anton:linear:8e:2000}.

Le nombre d'opérations de la décomposition $LU$ est du même ordre que
les méthodes d'élimination. Par contre, on remarquera que la
factorisation est tout à fait indépendante du vecteur $\mat{b}$. Une
fois la factorisation connue, on peut donc résoudre plusieurs systèmes
d'équations différents utilisant tous la même matrice de coefficients
$\mat{A}$ sans devoir répéter une grande partie des calculs.

\begin{rem}
  Plusieurs progiciels mathématiques ont recours à LAPACK\footnote{%
    \emph{Linear Algebra PACKage};
    \url{http://www.netlib.org/lapack}.} %
  pour les calculs d'algèbre linéaire numérique. Cette bibliothèque
  est elle-même basée sur LINPACK, dont les origines remontent aux
  années 1970. La bibliothèque est écrite en Fortran, un autre très
  ancien langage de programmation, toujours utilisé pour le calcul
  scientifique. D'ailleurs, au moins une particularité de ce langage
  demeure visible en R. En effet, c'est du Fortran que vient cette
  habitude de remplir les matrices par colonne\dots
\end{rem}

\chapter{Décomposition LU}
\label{chap:decomposition}

%%%
%%% Fichers de solutions et de réponses
%%%

\Opensolutionfile{reponses}[reponses-decomposition_lu]
\Opensolutionfile{solutions}[solutions-decomposition_lu]

\begin{Filesave}{reponses}
\bigskip
\section*{Réponses}

\end{Filesave}

\begin{Filesave}{solutions}
\section*{Chapitre \ref{chap:decomposition}}
\addcontentsline{toc}{section}{Chapitre \protect\ref{chap:decomposition}}

\end{Filesave}

%%%
%%% Début des exercices
%%%

\begin{exercice}
  Résoudre le système d'équations
  \begin{displaymath}
    \begin{array}[t]{*{2}{r@{\;}c@{\;}}r@{\;=\;}r}
      3x_1 &-& 6x_2 &-& 3x_3 &  -3 \\
      2x_1 & &      &+& 6x_3 & -22 \\
      -4x_1&+& 7x_2 &+& 4x_3 &   3
    \end{array}
  \end{displaymath}
  par la décomposition $LU$ sachant que
  \begin{displaymath}
    \begin{bmatrix}
      3 & -6 & -3 \\ 2 & 0 & 6 \\ -4 & 7 & 4
    \end{bmatrix} =
    \begin{bmatrix}
      3 & 0 & 0 \\ 2 & 4 & 0 \\ -4 & -1 & 2
    \end{bmatrix}
    \begin{bmatrix}
      1 & -2 & -1 \\ 0 & 1 & 2 \\ 0 & 0 & 1
    \end{bmatrix}.
  \end{displaymath}
  \begin{sol}
    La matrice
    \begin{displaymath}
      \mat{A} =
      \begin{bmatrix}
        3 & -6 & -3 \\ 2 & 0 & 6 \\ -4 & 7 & 4
      \end{bmatrix}
    \end{displaymath}
    est exprimée sous la forme $\mat{A} = \mat{L} \mat{U}$, où
    \begin{align*}
      \mat{L}
      &=
      \begin{bmatrix}
        3 & 0 & 0 \\ 2 & 4 & 0 \\ -4 & -1 & 2
      \end{bmatrix} \\
      \intertext{et}
      \mat{U}
      &=
      \begin{bmatrix}
        1 & -2 & -1 \\ 0 & 1 & 2 \\ 0 & 0 & 1
      \end{bmatrix}.
    \end{align*}
    Ainsi, le système d'équations $\mat{A} \mat{x} = \mat{b}$ (où
    $\mat{b} = (-3, -22, 3)^T$) peut être exprimé sous la forme
    $\mat{L} \mat{U} \mat{x} = \mat{b}$, soit $\mat{L} \mat{y} =
    \mat{b}$ et $\mat{U} \mat{x} = \mat{y}$. On résoud tout d'abord
    $\mat{L} \mat{y} = \mat{b}$ par simple substitution successive. On
    trouve
    \begin{align*}
      y_1 &= - 1 \\
      y_2 &= \frac{-22 - 2 y_1}{4} = -5 \\
      y_3 &= \frac{3 + 4 y_1 + y_2}{2} = -3.
    \end{align*}
    Par la suite, on résoud de même $\mat{U} \mat{x} = \mat{y}$, ce
    qui donne
    \begin{align*}
      x_3 &= -3 \\
      x_2 &= -5 - 2 x_3 = 1 \\
      x_1 &= -1 + 2 x_2 + x_3 = -2.
    \end{align*}
  \end{sol}
  \begin{rep}
    $x_1 = -2$, $x_2 = 1$, $x_3 = -3$
  \end{rep}
\end{exercice}

\begin{exercice}
  Soit
  \begin{align*}
    \mat{E}_1
    &= \begin{bmatrix} 1&0 \\ 2&3 \end{bmatrix} \\
    \mat{E}_2
    &= \begin{bmatrix} \frac{1}{3}&0 \\ 0&\frac{1}{3} \end{bmatrix} \\
    \intertext{et}
    \mat{A}
    &= \mat{E}_1^{-1} \mat{E}_2^{-1}
    \begin{bmatrix} 1&-2 \\ 0&1 \end{bmatrix}.
  \end{align*}
  Résoudre le système d'équations
  \begin{displaymath}
    \mat{A} \mat{x} = \begin{bmatrix} 0\\1 \end{bmatrix}
  \end{displaymath}
  par la décomposition $LU$.
  \begin{sol}
    On cherche tout d'abord des matrices triangulaires inférieure et
    supérieure $\mat{L}$ et $\mat{U}$, respectivement, tel que
    $\mat{A} = \mat{L} \mat{U}$. On nous donne dans l'énoncé
    \begin{displaymath}
      \mat{A} = \mat{E}_1^{-1} \mat{E}_2^{-1}
      \begin{bmatrix} 1&-2 \\ 0&1 \end{bmatrix},
    \end{displaymath}
    d'où
    \begin{displaymath}
      \mat{U} = \begin{bmatrix} 1&-2 \\ 0&1 \end{bmatrix},
    \end{displaymath}
    et $\mat{L} = \mat{E}_1^{-1} \mat{E}_2^{-1}$. Or,
    \begin{align*}
      \mat{E}_1^{-1}
      &= \frac{1}{3} \begin{bmatrix} 3&0 \\ -2&1 \end{bmatrix} \\
      \mat{E}_2^{-1}
      &= \begin{bmatrix} 3&0 \\ 0&3 \end{bmatrix} \\
      &= 3 \mat{I}
      \intertext{et, par conséquent,}
      \mat{L}
      &= \begin{bmatrix} 3&0 \\ -2&1 \end{bmatrix}.
    \end{align*}
    Pour résoudre par décomposition $LU$ le système d'équations
    $\mat{A} \mat{x} = \mat{L} \mat{U} \mat{x} = \mat{b}$, où $\mat{b}
    = (0, 1)$, on pose $\mat{U} \mat{x} = \mat{y}$ et résout d'abord
    $\mat{L} \mat{y} = \mat{b}$ par substitution. On a donc le système
    d'équations
    \begin{displaymath}
      \begin{bmatrix} 3&0 \\ -2&1 \end{bmatrix}
      \begin{bmatrix} y_1 \\ y_2 \end{bmatrix} =
      \begin{bmatrix} 0\\1 \end{bmatrix},
    \end{displaymath}
    dont la solution est $y_1 = 0$ et $y_2 = 1$. Par la suite, on a
    \begin{displaymath}
      \begin{bmatrix} 1&-2 \\ 0&1 \end{bmatrix}
      \begin{bmatrix} x_1 \\ x_2 \end{bmatrix} =
      \begin{bmatrix} 0\\1 \end{bmatrix},
    \end{displaymath}
    d'où, finalement, $x_1 = 2$ et $x_2 = 1$.
  \end{sol}
  \begin{rep}
    $\mat{x} = (2, 1)$
  \end{rep}
\end{exercice}

\Closesolutionfile{reponses}
\Closesolutionfile{solutions}

%%%
%%% Insérer les réponses
%%%
\input{reponses-decomposition_lu}


%%% Local Variables:
%%% mode: latex
%%% TeX-master: exercices_methodes_numeriques
%%% End:


%%% Local Variables:
%%% mode: latex
%%% TeX-master: "methodes_numeriques-partie_4"
%%% End:
