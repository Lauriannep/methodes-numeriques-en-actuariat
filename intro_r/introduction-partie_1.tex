\chapter*{Introduction}
\addcontentsline{toc}{chapter}{Introduction}
\markboth{Introduction}{Introduction}

Il existe de multiples ouvrages traitant de l'environnement
statistique R. Dans la majorité des cas, toutefois, le logiciel est
présenté dans le cadre d'applications statistiques spécifiques. Ce
document se concentre plutôt sur l'apprentissage du langage de
programmation sous-jacent aux diverses fonctions statistiques, langage
lui aussi nommé R.

Chaque chapitre présente en rafale plusieurs éléments de théorie avec
généralement peu d'exemples. La lecture d'un chapitre permet donc
d'acquérir rapidement plusieurs nouvelles connaissances sur le langage
R. Cependant, pour compléter son apprentissage, le lecteur devra aussi
étudier attentivement et, surtout, exécuter ligne par ligne le code R
fourni dans les sections d'exemples à la fin des chapitres (sauf un).
Ces sections d'exemples couvrent l'essentiel des concepts présentés
dans les chapitres et les complémentent souvent. L'étude de ces
sections fait partie intégrante de l'apprentissage du langage R.

Le code des sections d'exemples est disponible dans le site du cours.
Nous fournissons également des fichiers de sortie contenant les
résultats de chacune des expressions.

Un symbole de lecture vidéo dans la marge indique qu'une capsule vidéo
est disponible dans la %
\capsule{http://www.youtube.com/user/VincentGouletACT2002}{chaîne
  YouTube} %
du cours sur le sujet identifié comme un d'hyperlien.

Certains exemples et exercices font référence à des concepts de base
de la théorie des probabilités et des mathématiques financières. Les
contextes actuariels demeurent néanmoins peu nombreux et ne devraient
généralement pas dérouter le lecteur pour qui ces notions sont moins
familières. Les réponses de tous les exercices se trouvent en annexe.
En consultation électronique, le numéro d'un exercice est un hyperlien
vers sa réponse, et vice versa.

On trouvera également en annexe une brève introduction à l'éditeur de
texte GNU~Emacs et au mode ESS, ainsi qu'une présentation sur
l'administration d'une bibliothèque de packages R.

Je tiens à remercier M.~Mathieu Boudreault pour sa collaboration dans
la rédaction des exercices. Je remercie également Mmes~Marie-Pier
Laliberté et Véronique Tardif pour l'infographie des pages
couvertures.

%%% Local Variables:
%%% mode: latex
%%% TeX-master: "methodes_numeriques-partie_1"
%%% coding: utf-8
%%% End:
