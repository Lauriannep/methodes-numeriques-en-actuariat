%%% Copyright (C) 2018 Vincent Goulet
%%%
%%% Ce fichier fait partie du projet
%%% «Méthodes numériques en actuariat avec R»
%%% http://github.com/vigou3/methodes-numeriques-en-actuariat
%%%
%%% Cette création est mise à disposition selon le contrat
%%% Attribution-Partage dans les mêmes conditions 4.0
%%% International de Creative Commons.
%%% http://creativecommons.org/licenses/by-sa/4.0/

\chapter{Solutions des exercices}
\label{chap:solutions}
\markboth{Solutions des exercices}{Solutions des exercices}

\begingroup

%% Environnement Schunk simplifié pour l'affichage des réponses
\renewenvironment{Schunk}{%
  \setlength{\topsep}{0pt}
  \colorlet{shadecolor}{codebg}
  \begin{snugshade*}}%
  {\end{snugshade*}}
\input{solutions-generation}
\input{solutions-simulation}
\input{solutions-montecarlo}

\endgroup

%%% Local Variables:
%%% mode: latex
%%% TeX-engine: xetex
%%% TeX-master: "methodes-numeriques-en-actuariat_simulation"
%%% coding: utf-8
%%% End:
