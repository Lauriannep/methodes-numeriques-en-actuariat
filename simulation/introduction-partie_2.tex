\chapter*{Introduction}
\addcontentsline{toc}{chapter}{Introduction}
\markboth{Introduction}{Introduction}

La simulation stochastique est une technique utilisée dans un grand
nombre de domaines. On n'a qu'à penser aux simulations boursières qui
font l'objet d'un concours annuel, aux voitures qui sont d'abord
conçues sur ordinateur et soumises à des tests de collision virtuels,
ou encore aux prévisions météo quid ne en fait les résultats de
simulations de systèmes climatiques d'une grande complexité.

Toute simulation stochastique repose d'abord et avant tout sur une
source de nombres aléatoires de qualité. Comment en générer un grand
nombre rapidement et, surtout, comment s'assurer que les nombres
produits sont bien aléatoires? C'est un sujet d'une grande importance,
mais également fort complexe. Aussi ne ferons-nous que l'effleurer en
étudiant les techniques de base dans le
\autoref{chap:generation}.

En actuariat, nous avons habituellement besoin de nombres aléatoires
provenant d'une loi de probabilité non uniforme. Le
\autoref{chap:simulation} présente quelques algorithmes pour
transformer des nombres aléatoires uniformes en nombres non uniformes.
Évidemment, des outils informatiques sont aujourd'hui disponibles pour
générer facilement et rapidement des nombres aléatoires de diverses
lois de probabilité. Nous passons en revue les fonctionnalités de R et
de Excel à ce chapitre.

Enfin, cette partie du cours se termine au
\autoref{chap:montecarlo} par une application à première vue
inusitée de la simulation, soit le calcul d'intégrales définies par la
méthode dite Monte Carlo.

L'étude de ce document implique quelques allers-retours entre le texte
et les sections de code informatique présentes dans chaque chapitre.
Les sauts vers ces sections sont clairement indiqués dans le texte par
des mentions mises en évidence par le symbole {\ForwardToEnd}.

Un symbole de lecture vidéo dans la marge, tel que ci-contre, indique
qu'une capsule vidéo sur \capsule{le sujet} identifié par la marque de
soulignement est disponible dans le site du cours.

Chaque chapitre comporte des exercices. Les réponses de ceux-ci se
trouvent à la fin de chacun des chapitres et les solutions complètes,
en annexe. En consultation électronique, le numéro d'un exercice est
un hyperlien vers sa solution, et vice versa.

On trouvera également en annexe un bref exposé sur la planification
d'une simulation en R et des rappels sur la notion de transformation
de variables aléatoire.

Je tiens à souligner la précieuse collaboration de MM.~Mathieu
Boudreault, Sébastien Auclair et Louis-Philippe Pouliot lors de la
rédaction des exercices et des solutions.

%%% Local Variables:
%%% mode: latex
%%% TeX-master: "methodes_numeriques-partie_2"
%%% End:
