%%% Copyright (C) 2018 Vincent Goulet
%%%
%%% Ce fichier fait partie du projet
%%% «Méthodes numériques en actuariat avec R»
%%% http://github.com/vigou3/methodes-numeriques-en-actuariat
%%%
%%% Cette création est mise à disposition selon le contrat
%%% Attribution-Partage dans les mêmes conditions 4.0
%%% International de Creative Commons.
%%% http://creativecommons.org/licenses/by-sa/4.0/

\chapter*{Introduction}
\addcontentsline{toc}{chapter}{Introduction}
\markboth{Introduction}{Introduction}

La simulation stochastique est une technique utilisée dans un grand
nombre de domaines. Pensons seulement aux simulations boursières qui
font l'objet d'un concours annuel, aux voitures qui sont d'abord
conçues sur ordinateur et soumises à des tests de collisions virtuels,
ou encore aux prévisions météo qui ne en fait les résultats de
simulations de systèmes climatiques d'une grande complexité.

Toute simulation stochastique repose sur une source de nombres
aléatoires de qualité. Comment en générer un grand nombre rapidement
et, surtout, comment s'assurer que les nombres produits sont bien
aléatoires? C'est un sujet d'une grande importance, mais aussi fort
complexe. Nous nous contenterons donc de l'effleurer en étudiant les
techniques de base dans le \autoref{chap:generation}.

En actuariat, nous avons habituellement besoin de nombres aléatoires
provenant d'une loi de probabilité non uniforme. Le
\autoref{chap:simulation} présente quelques algorithmes pour
transformer des nombres aléatoires uniformes en nombres non uniformes.
Évidemment, des outils informatiques sont aujourd'hui disponibles pour
générer facilement et rapidement des nombres aléatoires de diverses
lois de probabilité. Nous passons en revue les fonctionnalités de R et
de Excel à ce chapitre.

Enfin, cette partie du cours se termine au
\autoref{chap:montecarlo} par une application à première vue
inusitée de la simulation, soit le calcul d'intégrales définies par la
méthode dite Monte Carlo.

Chaque chapitre propose un problème à résoudre au fil du texte.
L'énoncé du problème, les indications en cours de chapitre et la
solution complète se présentent dans des encadrés de couleur
contrastante et marqués des symboles {\faCogs}, {\faBolt} et
{\faLightbulbO}.

L'étude de ce document implique quelques allers-retours entre le texte
et les sections de code informatique présentes dans chaque chapitre.
Les sauts vers ces sections sont clairement indiqués dans le texte par
des mentions mises en évidence par le symbole {\faMapSigns}.

Tous les chapitres comportent des exercices. Les réponses de ceux-ci se
retrouvent à la fin de chacun des chapitres et les solutions complètes,
en annexe. En consultation électronique, le numéro d'un exercice est
un hyperlien vers sa solution, et vice versa.

Vous trouverez également en annexe un bref exposé sur la planification
d'une simulation en R et des rappels sur la transformation de
variables aléatoires.

Un symbole de lecture vidéo dans la marge indique qu'une capsule vidéo
est disponible dans la %
\capsule{http://www.youtube.com/user/VincentGouletACT2002}{chaîne
  YouTube} %
du cours sur le sujet en hyperlien.

Je tiens à souligner la précieuse collaboration de MM.~Mathieu
Boudreault, Sébastien Auclair et Louis-Philippe Pouliot lors de la
rédaction des exercices et des solutions.

%%% Local Variables:
%%% mode: latex
%%% TeX-engine: xetex
%%% TeX-master: "methodes-numeriques-en-actuariat_simulation"
%%% coding: utf-8
%%% End:
