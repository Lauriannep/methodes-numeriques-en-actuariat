\chapter{Décomposition LU}
\label{chap:decomposition}

%%%
%%% Fichers de solutions et de réponses
%%%

\Opensolutionfile{reponses}[reponses-decomposition_lu]
\Opensolutionfile{solutions}[solutions-decomposition_lu]

\begin{Filesave}{reponses}
\bigskip
\section*{Réponses}

\end{Filesave}

\begin{Filesave}{solutions}
\section*{Chapitre \ref{chap:decomposition}}
\addcontentsline{toc}{section}{Chapitre \protect\ref{chap:decomposition}}

\end{Filesave}

%%%
%%% Début des exercices
%%%

\begin{exercice}
  Résoudre le système d'équations
  \begin{displaymath}
    \begin{array}[t]{*{2}{r@{\;}c@{\;}}r@{\;=\;}r}
      3x_1 &-& 6x_2 &-& 3x_3 &  -3 \\
      2x_1 & &      &+& 6x_3 & -22 \\
      -4x_1&+& 7x_2 &+& 4x_3 &   3
    \end{array}
  \end{displaymath}
  par la décomposition $LU$ sachant que
  \begin{displaymath}
    \begin{bmatrix}
      3 & -6 & -3 \\ 2 & 0 & 6 \\ -4 & 7 & 4
    \end{bmatrix} =
    \begin{bmatrix}
      3 & 0 & 0 \\ 2 & 4 & 0 \\ -4 & -1 & 2
    \end{bmatrix}
    \begin{bmatrix}
      1 & -2 & -1 \\ 0 & 1 & 2 \\ 0 & 0 & 1
    \end{bmatrix}.
  \end{displaymath}
  \begin{sol}
    La matrice
    \begin{displaymath}
      \mat{A} =
      \begin{bmatrix}
        3 & -6 & -3 \\ 2 & 0 & 6 \\ -4 & 7 & 4
      \end{bmatrix}
    \end{displaymath}
    est exprimée sous la forme $\mat{A} = \mat{L} \mat{U}$, où
    \begin{align*}
      \mat{L}
      &=
      \begin{bmatrix}
        3 & 0 & 0 \\ 2 & 4 & 0 \\ -4 & -1 & 2
      \end{bmatrix} \\
      \intertext{et}
      \mat{U}
      &=
      \begin{bmatrix}
        1 & -2 & -1 \\ 0 & 1 & 2 \\ 0 & 0 & 1
      \end{bmatrix}.
    \end{align*}
    Ainsi, le système d'équations $\mat{A} \mat{x} = \mat{b}$ (où
    $\mat{b} = (-3, -22, 3)^T$) peut être exprimé sous la forme
    $\mat{L} \mat{U} \mat{x} = \mat{b}$, soit $\mat{L} \mat{y} =
    \mat{b}$ et $\mat{U} \mat{x} = \mat{y}$. On résoud tout d'abord
    $\mat{L} \mat{y} = \mat{b}$ par simple substitution successive. On
    trouve
    \begin{align*}
      y_1 &= - 1 \\
      y_2 &= \frac{-22 - 2 y_1}{4} = -5 \\
      y_3 &= \frac{3 + 4 y_1 + y_2}{2} = -3.
    \end{align*}
    Par la suite, on résoud de même $\mat{U} \mat{x} = \mat{y}$, ce
    qui donne
    \begin{align*}
      x_3 &= -3 \\
      x_2 &= -5 - 2 x_3 = 1 \\
      x_1 &= -1 + 2 x_2 + x_3 = -2.
    \end{align*}
  \end{sol}
  \begin{rep}
    $x_1 = -2$, $x_2 = 1$, $x_3 = -3$
  \end{rep}
\end{exercice}

\begin{exercice}
  Soit
  \begin{align*}
    \mat{E}_1
    &= \begin{bmatrix} 1&0 \\ 2&3 \end{bmatrix} \\
    \mat{E}_2
    &= \begin{bmatrix} \frac{1}{3}&0 \\ 0&\frac{1}{3} \end{bmatrix} \\
    \intertext{et}
    \mat{A}
    &= \mat{E}_1^{-1} \mat{E}_2^{-1}
    \begin{bmatrix} 1&-2 \\ 0&1 \end{bmatrix}.
  \end{align*}
  Résoudre le système d'équations
  \begin{displaymath}
    \mat{A} \mat{x} = \begin{bmatrix} 0\\1 \end{bmatrix}
  \end{displaymath}
  par la décomposition $LU$.
  \begin{sol}
    On cherche tout d'abord des matrices triangulaires inférieure et
    supérieure $\mat{L}$ et $\mat{U}$, respectivement, tel que
    $\mat{A} = \mat{L} \mat{U}$. On nous donne dans l'énoncé
    \begin{displaymath}
      \mat{A} = \mat{E}_1^{-1} \mat{E}_2^{-1}
      \begin{bmatrix} 1&-2 \\ 0&1 \end{bmatrix},
    \end{displaymath}
    d'où
    \begin{displaymath}
      \mat{U} = \begin{bmatrix} 1&-2 \\ 0&1 \end{bmatrix},
    \end{displaymath}
    et $\mat{L} = \mat{E}_1^{-1} \mat{E}_2^{-1}$. Or,
    \begin{align*}
      \mat{E}_1^{-1}
      &= \frac{1}{3} \begin{bmatrix} 3&0 \\ -2&1 \end{bmatrix} \\
      \mat{E}_2^{-1}
      &= \begin{bmatrix} 3&0 \\ 0&3 \end{bmatrix} \\
      &= 3 \mat{I}
      \intertext{et, par conséquent,}
      \mat{L}
      &= \begin{bmatrix} 3&0 \\ -2&1 \end{bmatrix}.
    \end{align*}
    Pour résoudre par décomposition $LU$ le système d'équations
    $\mat{A} \mat{x} = \mat{L} \mat{U} \mat{x} = \mat{b}$, où $\mat{b}
    = (0, 1)$, on pose $\mat{U} \mat{x} = \mat{y}$ et résout d'abord
    $\mat{L} \mat{y} = \mat{b}$ par substitution. On a donc le système
    d'équations
    \begin{displaymath}
      \begin{bmatrix} 3&0 \\ -2&1 \end{bmatrix}
      \begin{bmatrix} y_1 \\ y_2 \end{bmatrix} =
      \begin{bmatrix} 0\\1 \end{bmatrix},
    \end{displaymath}
    dont la solution est $y_1 = 0$ et $y_2 = 1$. Par la suite, on a
    \begin{displaymath}
      \begin{bmatrix} 1&-2 \\ 0&1 \end{bmatrix}
      \begin{bmatrix} x_1 \\ x_2 \end{bmatrix} =
      \begin{bmatrix} 0\\1 \end{bmatrix},
    \end{displaymath}
    d'où, finalement, $x_1 = 2$ et $x_2 = 1$.
  \end{sol}
  \begin{rep}
    $\mat{x} = (2, 1)$
  \end{rep}
\end{exercice}

\Closesolutionfile{reponses}
\Closesolutionfile{solutions}

%%%
%%% Insérer les réponses
%%%
\input{reponses-decomposition_lu}


%%% Local Variables:
%%% mode: latex
%%% TeX-master: exercices_methodes_numeriques
%%% End:
