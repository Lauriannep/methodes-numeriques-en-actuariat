\chapter*{Introduction}
\addcontentsline{toc}{chapter}{Introduction}
\markboth{Introduction}{Introduction}

Les ordinateurs ne savent pas compter. Ou, en fait, très peu. Ils ne
savent traiter que des $0$ et des $1$ et sont incapables de
représenter tous les nombres réels --- chose qu'un humain peut faire,
du moins conceptuellement. Cela signifie qu'à peu près tout calcul
effectué dans un ordinateur comporte une part d'erreur d'arrondi et de
troncature. Comme on ne souhaite généralement pas que cette erreur
devienne trop grande, il importe de connaître ses sources afin de la
diminuer le plus possible. C'est, entre autres choses, l'objet du
\autoref{chap:ordinateurs}.

Les procédures numériques pour résoudre des équations à une variable,
optimiser une fonction ou calculer une intégrale définie sont
aujourd'hui aisément accessibles dans une foule de logiciels à
connotation mathématique ou même dans une simple calculatrice. Or,
comment ces calculs sont-ils effectués, quels sont les algorithmes à
l'{\oe}uvre en arrière-scène? Le \autoref{chap:resolution} se
penche sur les méthodes de base de résolution d'équations à une
variable et le \autoref{chap:integration} sur celles
d'intégration numérique.

On présente également au \autoref{chap:resolution} les
principales fonctions d'optimisation disponibles dans Excel et dans R.

L'étude de ce document implique quelques allers-retours entre le texte
et les sections de code informatique présentes dans chaque chapitre.
Les sauts vers ces sections sont clairement indiqués dans le texte par
des mentions mises en évidence par le symbole {\ForwardToEnd}.

Chaque chapitre comporte des exercices. Les réponses de ceux-ci se
trouvent à la fin de chacun des chapitres et les solutions complètes,
en annexe.

Je tiens à souligner la précieuse collaboration de MM.~Mathieu
Boudreault, Sébastien Auclair et Louis-Philippe Pouliot lors de la
rédaction des exercices et des solutions. Je remercie également
Mmes~Marie-Pier Laliberté et Véronique Tardif pour l'infographie des
pages couvertures.

%%% Local Variables:
%%% mode: latex
%%% TeX-master: "methodes_numeriques-partie_3"
%%% End:
