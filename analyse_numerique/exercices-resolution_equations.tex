\chapter{Résolution d'équations à une variable}
\label{chap:resolution}

%%%
%%% Fichers de solutions et de réponses
%%%

\Opensolutionfile{reponses}[reponses-resolution_equations]
\Opensolutionfile{solutions}[solutions-resolution_equations]

\begin{Filesave}{reponses}
\bigskip
\section*{Réponses}

\end{Filesave}

\begin{Filesave}{solutions}
\section*{Chapitre \ref{chap:resolution}}
\addcontentsline{toc}{section}{Chapitre \protect\ref{chap:resolution}}

\end{Filesave}

%%%
%%% Début des exercices
%%%


\begin{exercice}
  Écrire des fonctions S pour effectuer les calculs des algorithmes de
  bissection, du point fixe, de Newton--Raphson et de la sécante.
  Outre les arguments communs à toutes les fonctions que sont le
  niveau de tolérance $\varepsilon$, le nombre maximal d'itérations
  $N_{\mathrm{max}}$ et une valeur booléenne spécifiant si les valeurs
  successives des itérations doivent être affichées à l'écran, les
  fonctions doivent compter les arguments mentionnés dans le tableau
  ci-dessous.
  \begin{center}
    \begin{tabular}{ll}
      \toprule
      Méthode d'approximation & Arguments de la fonction S \\
      \midrule
      Bissection & $f(x)$, $a$, $b$ \\
      Point fixe & $f(x)$, $x_0$ \\
      Newton--Raphson & $f(x)$, $f^\prime(x)$, $x_0$ \\
      Sécante & $f(x)$, $x_0$, $x_1$ \\
      \bottomrule
    \end{tabular}
  \end{center}
  \begin{sol}
    Se baser sur la fonction de point fixe \texttt{fp} présentée au
    chapitre 5 de \cite{Goulet_intro_S} pour composer les fonctions de
    bissection, de Newton--Raphson et de la sécante.
  \end{sol}
\end{exercice}

\begin{exercice}
  \label{ex:resolution:toutes}
  Trouver la solution des équations suivantes par les méthodes de
  bissection, de Newton-Raphson et de la sécante.
  \begin{enumerate}
  \item $x^3 - 2 x^2 - 5 = 0$ pour $1 \leq x \leq 4$
  \item $x^3 + 3 x^2 - 1 = 0$ pour $-4 \leq x \leq 0$
  \item $x - 2^{-x} = 0$ pour $0 \leq x \leq 1$
  \item $e^x + 2^{-x} + 2 \cos x - 6 = 0$ pour $1 \leq x \leq 2$
  \item $e^x - x^2 + 3x - 2 = 0$ pour $0 \leq x \leq 1$
  \end{enumerate}
  \begin{sol}
    Les solutions suivantes ont été obtenues à l'aide de nos fonctions
    de résolution d'équations à une variable. Les valeurs
    intermédiaires sont affichées pour montrer la convergence. La
    figure \ref{fig:resolution:toutes} contient les graphiques des cinq
    fonctions pour les intervalles mentionnés dans l'énoncé.
    \begin{figure}
      \centering
      \begin{minipage}{0.45\linewidth}
        \centering
\includegraphics{resolution_equations-002}
        \subcaption{$f(x) = x^3 - 2 x^2 - 5$}
      \end{minipage}
      \hfill
      \begin{minipage}{0.45\linewidth}
        \centering
\includegraphics{resolution_equations-003}
        \subcaption{$f(x) = x^3 + 3 x^2 - 1$}
      \end{minipage} \\
      \begin{minipage}{0.45\linewidth}
        \centering
\includegraphics{resolution_equations-004}
        \subcaption{$f(x) = x - 2^{-x}$}
      \end{minipage}
      \hfill
      \begin{minipage}{0.45\linewidth}
        \centering
\includegraphics{resolution_equations-005}
        \subcaption{$f(x) = e^x + 2^{-x} + 2 \cos x - 6$}
      \end{minipage} \\
      \begin{minipage}{0.45\linewidth}
        \centering
\includegraphics{resolution_equations-006}
        \subcaption{$f(x) = e^x - x^2 + 3x - 2$}
      \end{minipage}
      \caption{Fonctions de l'exercice
        \ref{chap:resolution}.\ref{ex:resolution:toutes}.}
      \label{fig:resolution:toutes}
    \end{figure}
    \begin{enumerate}
    \item La fonction n'a qu'une seule racine dans $[1, 4]$.
\begin{Schunk}
\begin{Sinput}
> f1 <- function(x) x^3 - 2 * x^2 - 5
> f1p <- function(x) 3 * x^2 - 4 * x
> bissection(f1, lower = 1, upper = 4, echo = TRUE)
\end{Sinput}
\begin{Soutput}
[1]  1.000  4.000  2.500 -1.875
[1] 2.5000 4.0000 3.2500 8.2031
[1] 2.5000 3.2500 2.8750 2.2324
[1]  2.500000  2.875000  2.687500 -0.034424
[1] 2.6875 2.8750 2.7812 1.0432
[1] 2.68750 2.78125 2.73438 0.49078
[1] 2.68750 2.73438 2.71094 0.22481
[1] 2.687500 2.710938 2.699219 0.094355
[1] 2.687500 2.699219 2.693359 0.029757
[1]  2.6875000  2.6933594  2.6904297 -0.0023855
[1] 2.690430 2.693359 2.691895 0.013673
[1] 2.6904297 2.6918945 2.6911621 0.0056403
[1] 2.6904297 2.6911621 2.6907959 0.0016266
[1]  2.69042969  2.69079590  2.69061279 -0.00037968
[1] 2.6906128 2.6907959 2.6907043 0.0006234
[1] 2.69061279 2.69070435 2.69065857 0.00012185
[1]  2.69061279  2.69065857  2.69063568 -0.00012892
[1]  2.6906e+00  2.6907e+00  2.6906e+00 -3.5365e-06
[1] 2.6906e+00 2.6907e+00 2.6907e+00 5.9155e-05
[1] 2.6906e+00 2.6907e+00 2.6906e+00 2.7809e-05
[1] 2.6906e+00 2.6906e+00 2.6906e+00 1.2136e-05
$root
[1] 2.6906

$nb.iter
[1] 21
\end{Soutput}
\begin{Sinput}
> nr(f1, f1p, start = 2.5, echo = TRUE)
\end{Sinput}
\begin{Soutput}
[1] 2.7143
[1] 2.691
[1] 2.6906
$root
[1] 2.6906

$nb.iter
[1] 3
\end{Soutput}
\begin{Sinput}
> secante(f1, start0 = 1, start1 = 4, echo = TRUE)
\end{Sinput}
\begin{Soutput}
[1] 1.5455
[1] 1.9969
[1] 4.1051
[1] 2.2947
[1] 2.4787
[1] 2.7514
[1] 2.6831
[1] 2.6904
[1] 2.6906
$root
[1] 2.6906

$nb.iter
[1] 9
\end{Soutput}
\end{Schunk}
    \item La fonction possède deux racines dans $[-4, 0]$. La
      convergence se fera vers l'une ou l'autre selon la position
      de la ou des valeurs de départ par rapport à l'extremum de la
      fonction dans l'intervalle. Ici, il s'agit d'un maximum $x =
      -2$. Ainsi, avec des valeurs de départ inférieures au maximum,
      on trouve la première racine:
\begin{Schunk}
\begin{Sinput}
> f2 <- function(x) x^3 + 3 * x^2 - 1
> f2p <- function(x) 3 * x^2 + 6 * x
> bissection(f2, lower = -3, upper = -2.8, echo = TRUE)
\end{Sinput}
\begin{Soutput}
[1] -3.000 -2.800 -2.900 -0.159
[1] -2.90000 -2.80000 -2.85000  0.21838
[1] -2.900000 -2.850000 -2.875000  0.033203
[1] -2.900000 -2.875000 -2.887500 -0.062014
[1] -2.887500 -2.875000 -2.881250 -0.014185
[1] -2.8812500 -2.8750000 -2.8781250  0.0095642
[1] -2.8812500 -2.8781250 -2.8796875 -0.0022966
[1] -2.8796875 -2.8781250 -2.8789062  0.0036373
[1] -2.87968750 -2.87890625 -2.87929688  0.00067121
[1] -2.87968750 -2.87929688 -2.87949219 -0.00081245
[1] -2.8795e+00 -2.8793e+00 -2.8794e+00 -7.0567e-05
[1] -2.87939453 -2.87929688 -2.87934570  0.00030034
[1] -2.87939453 -2.87934570 -2.87937012  0.00011489
[1] -2.8794e+00 -2.8794e+00 -2.8794e+00  2.2161e-05
[1] -2.8794e+00 -2.8794e+00 -2.8794e+00 -2.4203e-05
[1] -2.8794e+00 -2.8794e+00 -2.8794e+00 -1.0210e-06
[1] -2.87938538 -2.87938232 -2.87938385  0.00001057
$root
[1] -2.8794

$nb.iter
[1] 17
\end{Soutput}
\begin{Sinput}
> nr(f2, f2p, start = -3, echo = TRUE)
\end{Sinput}
\begin{Soutput}
[1] -2.8889
[1] -2.8795
[1] -2.8794
$root
[1] -2.8794

$nb.iter
[1] 3
\end{Soutput}
\begin{Sinput}
> secante(f2, start0 = -3, start1 = -2, echo = TRUE)
\end{Sinput}
\begin{Soutput}
[1] -2.75
[1] -3.0667
[1] -2.862
[1] -2.8772
[1] -2.8794
[1] -2.8794
$root
[1] -2.8794

$nb.iter
[1] 6
\end{Soutput}
\end{Schunk}
      Pour trouver la seconde racine, on utilise des valeurs de départ
      supérieures au maximum:
\begin{Schunk}
\begin{Sinput}
> bissection(f2, lower = -1, upper = 0.5, echo = TRUE)
\end{Sinput}
\begin{Soutput}
[1] -1.00000  0.50000 -0.25000 -0.82812
[1] -1.000000 -0.250000 -0.625000 -0.072266
[1] -1.00000 -0.62500 -0.81250  0.44409
[1] -0.81250 -0.62500 -0.71875  0.17850
[1] -0.718750 -0.625000 -0.671875  0.050953
[1] -0.671875 -0.625000 -0.648438 -0.011236
[1] -0.671875 -0.648438 -0.660156  0.019719
[1] -0.6601562 -0.6484375 -0.6542969  0.0042058
[1] -0.6542969 -0.6484375 -0.6513672 -0.0035239
[1] -0.65429688 -0.65136719 -0.65283203  0.00033872
[1] -0.6528320 -0.6513672 -0.6520996 -0.0015932
[1] -0.65283203 -0.65209961 -0.65246582 -0.00062736
[1] -0.65283203 -0.65246582 -0.65264893 -0.00014435
[1] -6.5283e-01 -6.5265e-01 -6.5274e-01  9.7175e-05
[1] -6.5274e-01 -6.5265e-01 -6.5269e-01 -2.3592e-05
[1] -6.5274e-01 -6.5269e-01 -6.5272e-01  3.6791e-05
[1] -6.5272e-01 -6.5269e-01 -6.5271e-01  6.5995e-06
[1] -6.5271e-01 -6.5269e-01 -6.5270e-01 -8.4961e-06
[1] -6.5271e-01 -6.5270e-01 -6.5270e-01 -9.4828e-07
[1] -6.5271e-01 -6.5270e-01 -6.5270e-01  2.8256e-06
[1] -6.5270e-01 -6.5270e-01 -6.5270e-01  9.3867e-07
[1] -6.527e-01 -6.527e-01 -6.527e-01 -4.804e-09
$root
[1] -0.6527

$nb.iter
[1] 22
\end{Soutput}
\begin{Sinput}
> nr(f2, f2p, start = -1, echo = TRUE)
\end{Sinput}
\begin{Soutput}
[1] -0.66667
[1] -0.65278
[1] -0.6527
$root
[1] -0.6527

$nb.iter
[1] 3
\end{Soutput}
\begin{Sinput}
> secante(f2, start0 = -2, start1 = -1, echo = TRUE)
\end{Sinput}
\begin{Soutput}
[1] -0.5
[1] -0.63636
[1] -0.65394
[1] -0.6527
[1] -0.6527
$root
[1] -0.6527

$nb.iter
[1] 5
\end{Soutput}
\end{Schunk}
      On remarquera que les deux valeurs de départ de la méthode de la
      sécante n'ont pas à se trouver de part et d'autre de la racine.
    \item La fonction n'a qu'une seule racine dans $[0, 1]$ et elle
      est légèrement supérieure à $0,6$.
\begin{Schunk}
\begin{Sinput}
> f3 <- function(x) x - 2^(-x)
> f3p <- function(x) 1 + log(2) * 2^(-x)
> bissection(f3, lower = 0.6, upper = 0.65, echo = TRUE)
\end{Sinput}
\begin{Soutput}
[1]  0.60000  0.65000  0.62500 -0.02342
[1]  0.6250000  0.6500000  0.6375000 -0.0053259
[1] 0.6375000 0.6500000 0.6437500 0.0037029
[1]  0.63750000  0.64375000  0.64062500 -0.00081001
[1] 0.6406250 0.6437500 0.6421875 0.0014468
[1] 0.6406250 0.6421875 0.6414062 0.0003185
[1]  0.64062500  0.64140625  0.64101562 -0.00024573
[1] 0.64101562 0.64140625 0.64121094 0.00003639
[1]  0.64101562  0.64121094  0.64111328 -0.00010467
[1]  0.64111328  0.64121094  0.64116211 -0.00003414
[1] 6.4116e-01 6.4121e-01 6.4119e-01 1.1251e-06
[1]  6.4116e-01  6.4119e-01  6.4117e-01 -1.6507e-05
[1]  6.4117e-01  6.4119e-01  6.4118e-01 -7.6910e-06
[1]  6.4118e-01  6.4119e-01  6.4118e-01 -3.2830e-06
[1]  6.4118e-01  6.4119e-01  6.4118e-01 -1.0789e-06
[1] 6.4118e-01 6.4119e-01 6.4119e-01 2.3101e-08
[1]  6.4118e-01  6.4119e-01  6.4119e-01 -5.2791e-07
$root
[1] 0.64119

$nb.iter
[1] 17
\end{Soutput}
\begin{Sinput}
> nr(f3, f3p, start = 0.6, echo = TRUE)
\end{Sinput}
\begin{Soutput}
[1] 0.641
[1] 0.64119
$root
[1] 0.64119

$nb.iter
[1] 2
\end{Soutput}
\begin{Sinput}
> secante(f3, start0 = 0.6, start1 = 0.65, echo = TRUE)
\end{Sinput}
\begin{Soutput}
[1] 0.64122
[1] 0.64119
$root
[1] 0.64119

$nb.iter
[1] 2
\end{Soutput}
\end{Schunk}
    \item La fonction n'a qu'une seule racine dans $[1, 2]$ et elle
      est légèrement supérieure à $1,8$.
\begin{Schunk}
\begin{Sinput}
> f4 <- function(x) exp(x) + 2^(-x) + 2 * cos(x) - 6
> f4p <- function(x) exp(x) - 2^(-x) * log(2) - 2 * sin(x)
> bissection(f4, lower = 1.8, upper = 1.85, echo = TRUE)
\end{Sinput}
\begin{Soutput}
[1]  1.800000  1.850000  1.825000 -0.017913
[1] 1.825000 1.850000 1.837500 0.033517
[1] 1.825000 1.837500 1.831250 0.007667
[1]  1.8250000  1.8312500  1.8281250 -0.0051567
[1] 1.8281250 1.8312500 1.8296875 0.0012468
[1]  1.8281250  1.8296875  1.8289063 -0.0019571
[1]  1.82890625  1.82968750  1.82929688 -0.00035568
[1] 1.8292969 1.8296875 1.8294922 0.0004454
[1] 1.8293e+00 1.8295e+00 1.8294e+00 4.4827e-05
[1]  1.82929688  1.82939453  1.82934570 -0.00015544
[1]  1.8293e+00  1.8294e+00  1.8294e+00 -5.5307e-05
[1]  1.8294e+00  1.8294e+00  1.8294e+00 -5.2405e-06
[1] 1.8294e+00 1.8294e+00 1.8294e+00 1.9793e-05
[1] 1.8294e+00 1.8294e+00 1.8294e+00 7.2762e-06
[1] 1.8294e+00 1.8294e+00 1.8294e+00 1.0178e-06
$root
[1] 1.8294

$nb.iter
[1] 15
\end{Soutput}
\begin{Sinput}
> nr(f4, f4p, start = 1.8, echo = TRUE)
\end{Sinput}
\begin{Soutput}
[1] 1.8301
[1] 1.8294
$root
[1] 1.8294

$nb.iter
[1] 2
\end{Soutput}
\begin{Sinput}
> secante(f4, start0 = 1.8, start1 = 1.85, echo = TRUE)
\end{Sinput}
\begin{Soutput}
[1] 1.8289
[1] 1.8294
[1] 1.8294
$root
[1] 1.8294

$nb.iter
[1] 3
\end{Soutput}
\end{Schunk}
    \item Encore ici, la fonction n'a qu'une seule racine dans
      l'intervalle mentionné et elle se situe autour de $0,25$.
\begin{Schunk}
\begin{Sinput}
> f5 <- function(x) exp(x) - x^2 + 3 * x - 2
> f5p <- function(x) exp(x) - 2 * x + 3
> bissection(f5, lower = 0.24, upper = 0.26, echo = TRUE)
\end{Sinput}
\begin{Soutput}
[1]  0.240000  0.260000  0.250000 -0.028475
[1]  0.2500000  0.2600000  0.2550000 -0.0095634
[1]  0.25500000  0.26000000  0.25750000 -0.00011444
[1] 0.2575000 0.2600000 0.2587500 0.0046084
[1] 0.2575000 0.2587500 0.2581250 0.0022471
[1] 0.2575000 0.2581250 0.2578125 0.0010664
[1] 0.25750000 0.25781250 0.25765625 0.00047597
[1] 0.25750000 0.25765625 0.25757812 0.00018077
[1] 2.5750e-01 2.5758e-01 2.5754e-01 3.3166e-05
[1]  2.5750e-01  2.5754e-01  2.5752e-01 -4.0637e-05
[1]  2.5752e-01  2.5754e-01  2.5753e-01 -3.7355e-06
[1] 2.5753e-01 2.5754e-01 2.5753e-01 1.4715e-05
[1] 2.5753e-01 2.5753e-01 2.5753e-01 5.4898e-06
[1] 2.5753e-01 2.5753e-01 2.5753e-01 8.7717e-07
[1]  2.5753e-01  2.5753e-01  2.5753e-01 -1.4291e-06
[1]  2.5753e-01  2.5753e-01  2.5753e-01 -2.7598e-07
[1] 2.5753e-01 2.5753e-01 2.5753e-01 3.0059e-07
$root
[1] 0.25753

$nb.iter
[1] 17
\end{Soutput}
\begin{Sinput}
> nr(f5, f5p, start = 0.26, echo = TRUE)
\end{Sinput}
\begin{Soutput}
[1] 0.25753
[1] 0.25753
$root
[1] 0.25753

$nb.iter
[1] 2
\end{Soutput}
\begin{Sinput}
> secante(f5, start0 = 0.24, start1 = 0.26, echo = TRUE)
\end{Sinput}
\begin{Soutput}
[1] 0.25753
[1] 0.25753
$root
[1] 0.25753

$nb.iter
[1] 2
\end{Soutput}
\end{Schunk}
    \end{enumerate}
  \end{sol}
\end{exercice}

\begin{exercice}
  \label{ex:resolution:sqrt(2)}
  Déterminer la valeur numérique de $\sqrt{2}$ à l'aide de la méthode
  de bissection dans l'intervalle $[0, 2] $ avec 10 itérations.
  Comparer avec la vraie valeur.
  \begin{sol}
    On trouve la racine de $f(x) = x^2 - 2$ dans l'intervalle $[0, 2]$
    par la méthode de bissection avec un maximum de 10 itérations. Le
    tableau \ref{tab:resolution:sqrt(2)} contient les valeurs
    successives de $a$, $b$, $x = (a + b)/2$ et $f(x)$.
    \begin{table}[t]
      \centering
      \caption{\normalfont Valeurs successives de la méthode de
        bissection pour l'exercice
        \ref{chap:resolution}.\ref{ex:resolution:sqrt(2)}.}
      \label{tab:resolution:sqrt(2)}
      \begin{tabular}{rllll}
        \toprule
        \multicolumn{1}{c}{$n$} &
        \multicolumn{1}{c}{$a_n$} &
        \multicolumn{1}{c}{$b_n$} &
        \multicolumn{1}{c}{$x_n$} &
        \multicolumn{1}{c}{$f(x_n)$} \\
        \midrule
        1 & 0 & 2 & 1 & $-1$ \\
        2 & 1,00 & 2,00 & 1,50 & 0,25 \\
        3 & 1,0000 &  1,5000 &  1,2500 & $-0,4375$ \\
        4 & 1,250000 &  1,500000 &  1,375000 & $-0,109375$ \\
        5 & 1,37500000 & 1,50000000 & 1,43750000 & 0,06640625 \\
        6 & 1,37500000 &  1,43750000 &  1,40625000 & $-0,02246094$ \\
        7 & 1,40625000 & 1,43750000 & 1,42187500 & 0,02172852 \\
        8 & 1,4062500000 &  1,4218750000 &  1,4140625000 & $-0,0004272461$ \\
        9 & 1,41406250 & 1,42187500 & 1,41796875 & 0,01063538 \\
        10 & 1,41406250 & 1,41796875 & 1,41601562 & 0,00510025 \\
        \bottomrule
      \end{tabular}
    \end{table}
    On obtient donc une réponse de $1,41601562$, alors que la vraie
    réponse est le nombre irrationnel $\sqrt{2} = 1,414214$.
  \end{sol}
\end{exercice}

\begin{exercice}
  Soit $\{x_n\}$ une suite définie par
  \begin{displaymath}
    x_n = \sum_{k=1}^n \frac{1}{k}.
  \end{displaymath}
  Démontrer que $\lim_{n \rightarrow \infty} (x_n - x_{n-1}) = 0$,
  mais que la suite diverge néanmoins. Ceci illustre que l'erreur
  absolue peut être un mauvais critère d'arrêt dans les méthodes
  numériques.
  \begin{sol}
    On a que
    \begin{align*}
      x_n - x_{n - 1}
      &= \sum_{k=1}^n \frac{1}{k} - \sum_{k=1}^{n - 1} \frac{1}{k} \\
      &= \frac{1}{n},
    \end{align*}
    d'où $\lim_{n \rightarrow \infty} (x_n - x_{n-1}) = \lim_{n
      \rightarrow \infty} n^{-1} = 0$. Or, $\sum_{k=1}^n k^{-1}$ est
    la série harmonique qui est connue pour diverger. On peut, par
    exemple, justifier ceci par le fait que l'intégrale
    \begin{displaymath}
      \int_1^\infty \frac{1}{x}\, dx
    \end{displaymath}
    diverge elle-même. Cet exercice illustre donc que le critère $|x_n
    - x_{n - 1}| < \varepsilon$ peut être satisfait même pour une
    série divergente.
  \end{sol}
\end{exercice}

\newpage

\begin{exercice}
  Considérer la fonction $g(x) = 3^{-x}$ sur l'intervalle $[0,
  1]$.
  \begin{enumerate}
  \item Vérifier si les hypothèses du théorème 5.1 des notes de cours
    quant à l'existence et l'unicité d'un point fixe dans $[0, 1]$
    sont satisfaites.
  \item Déterminer graphiquement l'existence et l'unicité d'un point
    fixe de $g(x)$ dans $[0, 1]$ puis, le cas échéant, calculer ce
    point fixe.
  \end{enumerate}
  \begin{sol}
    \begin{enumerate}
    \item On considère la fonction $g(x) = 3^{-x}$ dans l'intervalle
      $[0, 1]$. En premier lieu, $1/3 \leq 3^{-x} \leq 1$ pour $0 \leq
      x \leq 1$, donc $g(x) \in [0, 1]$. De plus, $g^\prime(x) = -
      3^{-x - 1} (\ln 3)$, d'où
      \begin{displaymath}
        |g^\prime(x)| = \frac{\ln 3}{3^{x + 1}} \leq \ln 3 < 1, \quad
        0 \leq x \leq 1.
      \end{displaymath}
      Par le théorème 5.1 des notes de cours, la fonction $g$ possède
      un point fixe unique dans l'intervalle $[0, 1]$.
    \item Le graphique de la fonction $g$ se trouve à la figure
      \ref{fig:resolution:3^-x}. On constate que la fonction a effectivement un
      point fixe unique dans l'intervalle $[0, 1]$ et que celui-ci se
      trouve près de $x = \frac{1}{2}$. On peut donc utiliser cette
      valeur comme point de départ de l'algorithme du point fixe:
\begin{Schunk}
\begin{Sinput}
> pointfixe(function(x) 3^(-x), start = 0.5)
\end{Sinput}
\begin{Soutput}
$fixed.point
[1] 0.54781

$nb.iter
[1] 24
\end{Soutput}
\end{Schunk}
      On remarque que la réponse est indépendante de la valeur de
      départ:
\begin{Schunk}
\begin{Sinput}
> pointfixe(function(x) 3^(-x), start = 0)
\end{Sinput}
\begin{Soutput}
$fixed.point
[1] 0.54781

$nb.iter
[1] 29
\end{Soutput}
\begin{Sinput}
> pointfixe(function(x) 3^(-x), start = 1)
\end{Sinput}
\begin{Soutput}
$fixed.point
[1] 0.54781

$nb.iter
[1] 28
\end{Soutput}
\begin{Sinput}
> pointfixe(function(x) 3^(-x), start = 2)
\end{Sinput}
\begin{Soutput}
$fixed.point
[1] 0.54781

$nb.iter
[1] 29
\end{Soutput}
\end{Schunk}
      \begin{figure}
        \centering
\includegraphics{resolution_equations-017}
        \caption{Fonction $g(x) = 3^{-x}$ dans $[0, 1]$.}
        \label{fig:resolution:3^-x}
      \end{figure}
    \end{enumerate}
  \end{sol}
\end{exercice}

\begin{exercice}
  Vérifier que les cinq fonctions $g_1, \dots, g_5$ de l'exemple 5.4
  des notes de cours ont toutes un point fixe en $x^*$ lorsque $f(x^*)
  = 0$, où $f(x) = x^3 + 4x^2 - 10$.
  \begin{sol}
    Les cinq fonctions sont les suivantes:
    \begin{align*}
      g_1(x) &= x - x^3 - 4 x^2 + 10  \\
      g_2(x) &= \left( \frac{10}{x} - 4x \right)^{1/2} \\
      g_3(x) &= \frac{1}{2} (10 - x^3)^{1/2} \\
      g_4(x) &= \left( \frac{10}{4 + x} \right)^{1/2} \\
      g_5(x) &= x - \frac{x^3 + 4 x^2 - 10}{3x^2 + 8x}.
    \end{align*}
    Soit $f(x) = x^3 + 4x^2 - 10$. On peut réécrire l'équation $f(x) =
    0$ de différentes façons:
    \begin{enumerate}
    \item $x = x - f(x)$;
    \item $x^3 = 10 - 4x^2 \Rightarrow x^2 = 10/x - 4x \Rightarrow x =
      (10/x - 4x)^{1/2}$;
    \item $4x^2 = 10 - x^3 \Rightarrow x^2 = (10 - x^3)/4 \Rightarrow
      x = \frac{1}{2}(10 - x^3)^{1/2}$;
    \item $4x^2 +  x^3 = 10 \Rightarrow x^2 = 10/(4 + x) \Rightarrow x
      = (10/(4 + x))^{1/2}$;
    \item $x = x - f(x)/f^\prime(x)$.
    \end{enumerate}
    Les fonctions $g_1$ à $g_5$ correspondent, dans l'ordre, au côté
    droit de chacune des équations ci-dessus. On remarquera que la
    fonction $g_5$ est celle préconisée par la méthode de
    Newton--Raphson --- et celle qui converge le plus rapidement.
  \end{sol}
\end{exercice}

\begin{exercice}
  Soit $g(x) = 4x^2 - 14x + 14$. Pour quels intervalles de valeurs de
  départ la procédure de point fixe converge-t-elle et diverge-t-elle?
  \begin{sol}
    La fonction $g(x) = 4x^2 - 14x + 14$ est une parabole ayant deux
    points fixes. On observe graphiquement que l'un se trouve en $x =
    2$ et que $g(1,5) = g(2)$. Or, pour toute valeur de départ $x_0 <
    1,5$ de l'algorithme du point fixe, on constate que les valeurs de
    $x_1, x_2, \dots$ se retrouveront de plus en plus loin sur la
    branche droite de la parabole. Il en va de même pour tout $x_0 >
    2$. La procédure diverge donc pour de telles valeurs de départ. En
    revanche, il est facile de vérifier graphiquement que la procédure
    converge pour toute valeur de départ dans l'intervalle $[1,5,\,
    2]$. Si $x_0 = 1,5$, on obtient le point fixe $x = 2$ après une
    seule itération.
  \end{sol}
  \begin{rep}
    Converge pour $x_0 \in [1,5, 2]$; diverge pour $x_0 \leq 1,5$ et
    $x_0 > 2$
  \end{rep}
\end{exercice}

\begin{exercice}
  \label{ex:resolution:racinecubique}
  Les trois fonctions ci-dessous sont toutes des candidates pour faire
  l'approximation, par la méthode du point fixe, de $\sqrt[3]{21}$:
  \begin{align*}
    g_1(x) &= \frac{20 x + 21x^{-2}}{21} \\
    g_2(x) &= x - \frac{x^3 - 21}{3 x^2} \\
    g_3(x) &= x - \frac{x^4 - 21 x}{x^2 - 21}.
  \end{align*}
  Classer ces fonctions en ordre décroissant de vitesse de convergence
  de l'algorithme du point fixe. \emph{Astuce}: comparer les valeurs
  des dérivées autour du point fixe.
  \begin{sol}
    En premier lieu, on remarque que les fonctions $g_1$, $g_2$ et
    $g_3$ sont développées à partir de l'équation $x^3 = 21$ pour
    calculer $\sqrt[3]{21}$ par la méthode du point fixe. La figure
    \ref{fig:resolution:racinecubique} présente ces trois fonctions.
    \begin{figure}
      \centering
      \begin{minipage}[t]{0.32\linewidth}
        \centering
\includegraphics{resolution_equations-018}
        \subcaption{\itshape Fonction $g_1(x)$}
      \end{minipage}
      \hfill
      \begin{minipage}[t]{0.32\linewidth}
        \centering
\includegraphics{resolution_equations-019}
        \subcaption{\itshape Fonction $g_2(x)$}
      \end{minipage}
      \hfill
      \begin{minipage}[t]{0.32\linewidth}
        \centering
\includegraphics{resolution_equations-020}
        \subcaption{\itshape Fonction $g_3(x)$}
      \end{minipage}
      \caption{Fonctions de l'exercice
        \ref{chap:resolution}.\ref{ex:resolution:racinecubique}.}
      \label{fig:resolution:racinecubique}
    \end{figure}
    On a
    \begin{align*}
      g_1^\prime(x)
      &= \frac{20}{21} - \frac{2}{x^3} \\
      g_2^\prime(x)
      &= \frac{2}{3} - \frac{14}{x^3} \\
      g_3^\prime(x)
      &= 1 - \frac{2x^5 - 84 x^3 + 21 x^2 + 441}{(x^2 - 21)^2}.
    \end{align*}
    Considérons l'intervalle $[2,5,\, 3]$ autour du point fixe. On a
    que $|g_2^\prime(x)| < |g_1(x)|$ pour tout $x$ dans cet
    intervalle, donc la procédure du point fixe converge plus
    rapidement pour $g_2$ (la fonction de la méthode de
    Newton--Raphson). Cela se vérifie aisément sur les graphiques de
    la figure \ref{fig:resolution:racinecubique}. Toujours à l'aide
    des graphiques, on voit que la procédure diverge avec la fonction
    $g_3$.
  \end{sol}
  \begin{rep}
    $g_2$, $g_1$; $g_3$ ne converge pas
  \end{rep}
\end{exercice}

\begin{exercice}
  Démontrer, à l'aide du théorème 5.1 des notes de cours, que la
  fonction $g(x) = 2^{-x}$ possède un point fixe unique dans
  l'intervalle $[\frac{1}{3}, 1]$. Calculer par la suite ce point fixe
  à l'aide de votre fonction S.
  \begin{sol}
    On considère la fonction $g(x) = 2^{-x}$ dans l'intervalle
    $[\frac{1}{3}, 1]$. En premier lieu, on a $g(x) \in [\frac{1}{2},
    1/\sqrt[3]{2}] \subset [\frac{1}{3}, 1]$ pour $x \in [\frac{1}{3},
    1]$. De plus, on a $g^\prime(x) = -2^{-x - 1} (\ln 2)$, d'où
    \begin{displaymath}
      |g^\prime(x)| = \frac{\ln 2}{2^{x + 1}} \leq \ln 2 < 1, \quad
      \frac{1}{3} \leq x \leq 1.
    \end{displaymath}
    Par le théorème 5.1 des notes de cours, la fonction $g$ possède
    donc un point fixe unique dans l'intervalle $[\frac{1}{3}, 1]$. La
    valeur de ce point fixe est
\begin{Schunk}
\begin{Sinput}
> pointfixe(function(x) 2^(-x), start = 2/3)
\end{Sinput}
\begin{Soutput}
$fixed.point
[1] 0.64119

$nb.iter
[1] 14
\end{Soutput}
\end{Schunk}
  \end{sol}
\end{exercice}

\begin{exercice}
  Vérifier graphiquement que le théorème 5.1 des notes de cours
  demeure valide si la condition $|g^\prime(x)| \leq k < 1$ est
  remplacée par $g^\prime(x) \leq k < 1$.
  \begin{sol}
    La condition $|g^\prime(x)| \leq k < 1$ du théorème 5.1 assure
    l'unicité d'un point fixe. Cette condition est nécessaire pour que
    la fonction $g(x)$ ne puisse repasser au-dessus de la droite $y =
    x$ après l'avoir déjà croisée une première fois pour créer un
    point fixe. Or, il est seulement nécessaire de limiter la
    croissance de la fonction --- soit les valeurs positives de sa
    pente. Peu importe la vitesse de décroissance de la fonction dans
    l'intervalle $[a, b]$ (mais toujours sujet à ce que $g(x) \in [a,
    b]$), la fonction ne peut croiser la droite $y = x$ qu'une seule
    fois. La condition $g^\prime(x) \leq k < 1$ est donc suffisante
    pour assurer l'unicité du point fixe dans $[a, b]$.
  \end{sol}
\end{exercice}

\begin{exercice}
  Utiliser la méthode de Newton--Raphson pour trouver le point sur la
  courbe $y = x^2$ le plus près du point $(1, 0)$. \emph{Astuce}:
  minimiser la distance entre le point $(1, 0)$ et le point $(x,
  x^2)$.
  \begin{sol}
    La distance entre deux points $(x_1, y_1)$ et $(x_2, y_2)$ de
    $\mathbb{R}^2$ est
    \begin{displaymath}
      d = \sqrt{(x_1 - x_2)^2 + (y_1 - y_2)^2}.
    \end{displaymath}
    Par conséquent, la distance entre le point $(x, x^2)$ et le point
    $(1, 0)$ est une fonction de $x$
    \begin{displaymath}
      d(x) = \sqrt{(x - 1)^2 + (x^2 - 0)^2} = \sqrt{x^4 + x^2 - 2x + 1}.
    \end{displaymath}
    On cherche la valeur de minimisant $d(x)$ ou, de manière
    équivalente, $d^2(x)$. On doit donc résoudre
    \begin{displaymath}
      \frac{d}{dx} d^2(x) = f(x) = 4x^3 + 2x - 2 = 0
    \end{displaymath}
    à l'aide de la méthode de Newton-Raphson. Cela requiert également
    \begin{displaymath}
      f^\prime(x) = 12 x^2 + 2.
    \end{displaymath}
    Le point sur la courbe $y = x^2$ le plus du point $(1, 0)$ est
    donc la limite de la suite
    \begin{displaymath}
      x_n = x_{n - 1} - \frac{4x_{n - 1}^3 + 2x_{n - 1} - 2}{12x_{n -
          1}^2 + 2}, \quad n = 1, 2, \dots,
    \end{displaymath}
    avec $x_0$ une valeur de départ «près» de la solution. On obtient
\begin{Schunk}
\begin{Sinput}
> nr(function(x) 4 * x^3 + 2 * x - 2, function(x) 12 * x^2 + 2, start = 0.6)
\end{Sinput}
\begin{Soutput}
$root
[1] 0.58975

$nb.iter
[1] 2
\end{Soutput}
\end{Schunk}

    On voit à la figure \ref{fig:resolution:proche} que la solution, disons le
    point $(x^*, (x^*)^2)$, est la projection du point $(1, 0)$ sur la
    courbe $y = x^2$; autrement dit, la droite passant par $(x^*,
    (x^*)^2)$ et $(1, 0)$ est perpenticulaire à la tangente de $y =
    x^2$ en $x = x^*$.
    \begin{figure}
      \centering
\includegraphics{resolution_equations-023}
      \caption{Point de $y = x^2$ le plus près du point $(1, 0)$.}
      \label{fig:resolution:proche}
    \end{figure}
  \end{sol}
  \begin{rep}
    $0,589755$
  \end{rep}
\end{exercice}

\begin{exercice}
  \label{ex:resolution:tri}
  Le taux de rendement interne d'une série de flux
  financiers $\{\mathit{CF}_t\}$ est le taux $i$ tel que
  \begin{displaymath}
    \sum_{t=0}^{n} \frac{\mathit{CF}_{t}}{(1 + i)^{t}} = 0.
  \end{displaymath}
  Écrire une fonction S permettant de calculer le taux de rendement
  interne d'une série de flux financiers quelconque à l'aide de la
  méthode de Newton--Raphson. Les arguments de la fonction sont un
  vecteur \texttt{CF} et un scalaire \texttt{erreur.max}. Au moins un
  élément de \texttt{CF} doit être négatif pour représenter une sortie
  de fonds. Dans tous les cas, utiliser $i = 0,05$ comme valeur de
  départ.
  \begin{sol}
    La fonction devra utiliser la méthode de Newton--Raphson pour
    trouver la racine de $f(i) = 0$, où
    \begin{align*}
      f(i)
      &= \sum_{t=0}^{n} \frac{\mathit{CF}_{t}}{(1 + i)^{t}} \\
      \intertext{et}
      f^\prime(i)
      &= - \sum_{t=0}^{n} \frac{t \mathit{CF}_{t}}{(1 + i)^{t + 1}}.
    \end{align*}
    La fonction \texttt{tri} de la figure \ref{fig:resolution:tri} calcule le
    taux de rendement interne d'un vecteur de flux financiers en
    utilisant notre fonction de Newton--Raphson \texttt{nr}.
    \begin{figure}
      \centering
      \begin{framed}
\begin{Scode}
tri <- function(CF, erreur.max = 1E-6)
{
    if (!any(CF < 0))
        stop("au moins un flux financier doit être
              négatif")

    t <- 0:(length(CF) - 1)

    f <- function(i) sum(CF/(1 + i)^t)
    fp <- function(i) sum((-CF * t)/((1 + i)^(t + 1)))

    nr(f, fp, start = 0.05, TOL = erreur.max)$root
}
\end{Scode}
      \end{framed}
      \caption{Fonction S de calcul du taux de rendement interne d'une
        série de flux financiers.}
      \label{fig:resolution:tri}
    \end{figure}
  \end{sol}
\end{exercice}

\begin{exercice}
  Refaire l'exercice \ref{chap:resolution}.\ref{ex:resolution:tri} en VBA en composant une fonction
  accessible dans une feuille Excel et comparer le résultat avec la
  fonction Excel \texttt{TRI()}.
  \begin{sol}
    La figure \ref{fig:resolution:tri2} contient le code d'une fonction VBA
    \texttt{tri2} effectuant le calcul du taux de rendement interne
    d'un ensemble de flux financiers. Ceux-ci sont passés en argument
    à la fonction par le bais d'une plage horizontale ou verticale de
    montants périodiques consécutifs. La procédure de Newton--Raphson
    est codée à même cette fonction. On peut vérifier facilement que
    le résultat de cette fonction est identique à celui de la fonction
    Excel \texttt{TRI()}.
    \begin{figure}
      \centering
      \begin{framed}
\begin{Scode}
Function tri2(flux As Range, Optional ErreurMax
              As Double = 0.000000001)
    Dim nrow As Integer, ncol As Integer, n As Integer
    Dim CF() As Double, f As Double, fp As Double,
        i As Double, it As Double

    nrow = flux.Rows.Count
    ncol = flux.Columns.Count
    n = IIf(nrow > 1, nrow, ncol) - 1
    ReDim CF(0 To n)

    For t = 0 To n
        CF(t) = flux.Cells(t + 1, 1)
    Next t

    If CF(0) > 0 Then
        tri2 = "Erreur"
        Exit Function
    End If

    i = 0.05
    Do
        f = 0
        fp = 0
        it = i
        For t = 0 To n
            f = f + CF(t) / (1 + i) ^ t
            fp = fp - t * CF(t) / (1 + i) ^ (t + 1)
        Next t
        i = i - f / fp
    Loop Until Abs(i - it) / i < ErreurMax
    tri2 = i
End Function
\end{Scode}
      \end{framed}
      \caption{Fonction VBA de calcul du taux de rendement interne d'une
        série de flux financiers.}
      \label{fig:resolution:tri2}
    \end{figure}
  \end{sol}
\end{exercice}

\begin{exercice}
  \label{ex:resolution:gamma}
  \begin{enumerate}
  \item Écrire une fonction S pour estimer par la méthode du maximum
    de vraisemblance le paramètre $\theta$ d'une loi gamma de
    paramètre de forme $\alpha = 3$ et de paramètre d'échelle $\theta
    = 1/\lambda$ --- de moyenne $3 \theta$, donc --- à partir d'un
    échantillon aléatoire $x_1, \dots, x_n$. \emph{Astuce}: maximiser
    la fonction de log-vraisemblance plutôt que la fonction de
    vraisemblance.
  \item Simuler 20 observations d'une loi gamma avec paramètre de
    forme $\alpha = 3$ et moyenne $\nombre{3000}$, puis estimer le
    paramètre d'échelle $\theta$ par le maximum de vraisemblance.
  \end{enumerate}
  \begin{sol}
    \begin{enumerate}
    \item On veut estimer par la méthode du maximum de vraisemblance
      le paramètre $\theta$ d'une loi Gamma$(3, 1/\theta)$, dont la
      fonction de densité de probabilité est
      \begin{displaymath}
        f(x; \theta) = \frac{1}{2 \theta^3}\, x^2 e^{-x/\theta}, \quad
        x > 0.
      \end{displaymath}
      La fonction de log-vraisemblance d'un échantillon aléatoire
      $x_1, \dots, x_n$ tiré de cette loi est
      \begin{align*}
        l(\theta)
        &= \sum_{i = 1}^n \ln f(x_i; \theta) \\
        &= \sum_{i = 1}^n
        \left(
          2 \ln x_i - \frac{x_i}{\theta} - \ln 2 - 3 \ln \theta
        \right) \\
        &= 2 \sum_{i = 1}^n \ln x_i - \frac{1}{\theta} \sum_{i = 1}^n
        x_i - n \ln 2 - 3n \ln \theta.
      \end{align*}
      On cherche le maximum de la fonction $l(\theta)$, soit la valeur
      de $\theta$ tel que
      \begin{displaymath}
        l^\prime(\theta) = \frac{1}{\theta^2} \sum_{i = 1}^n -
        \frac{3n}{\theta} = 0.
      \end{displaymath}
      Résoudre cette équation à l'aide de la méthode de
      Newton--Raphson requiert également
      \begin{displaymath}
        l^{\prime\prime}(\theta) = - \frac{2}{\theta^3} \sum_{i =
          1}^n x_i + \frac{3n}{\theta^2}.
      \end{displaymath}
      La figure \ref{fig:resolution:emv.gamma} présente une fonction S pour
      effectuer le calcul à l'aide de notre fonction de
      Newton--Raphson \texttt{nr}.  On remarquera que la somme des
      valeurs de l'échantillon --- qui ne change pas durant la
      procédure itérative --- est calculée une fois pour toute dès le
      début de la fonction.  De plus, on utilise comme valeur de
      départ l'estimateur des moments de $\theta$, $\bar{x}/3$.
      \begin{figure}
        \centering
        \begin{framed}
\begin{Scode}
emv.gamma <- function(x, erreur.max = 1E-6)
{
    n <- length(x)
    xsum <- sum(x)
    f <- function(theta) xsum/theta^2 -
         (3 * n)/theta
    fp <- function(theta) -2 * xsum / theta^3 +
          (3 * n)/theta^2
    nr(f, fp, xsum / n /3, TOL = erreur.max)$root
}
\end{Scode}
        \end{framed}
        \caption{Fonction S d'estimation du paramètre d'échelle d'une
          loi gamma par le maximum de vraisemblance.}
        \label{fig:resolution:emv.gamma}
      \end{figure}
    \item On a, par exemple,
\begin{Schunk}
\begin{Sinput}
> x <- rgamma(20, shape = 3, scale = 1000)
> emv.gamma(x)
\end{Sinput}
\begin{Soutput}
[1] 952.32
\end{Soutput}
\end{Schunk}
    \end{enumerate}
  \end{sol}
\end{exercice}

\begin{exercice}
  Refaire l'exercice \ref{chap:resolution}.\ref{ex:resolution:gamma} à
  l'aide d'une routine VBA. Comparer votre réponse avec celle obtenue
  à l'aide du Solveur de Excel.
  \begin{sol}
    La figure \ref{fig:resolution:emvGamma} présente le code VBA d'une fonction
    \texttt{emvGamma} très similaire à la fonction S de l'exercice
    précédent. Pour l'utiliser dans Excel, il suffit de lui passer en
    argument une plage (verticale) contenant un échantillon aléatoire
    d'une loi Gamma$(3, 1/\theta)$.
    \begin{figure}
      \centering
      \begin{framed}
\begin{Scode}
Function emvGamma(Data As Range, Optional ErreurMax
                  As Double = 0.000000001)
    Dim n As Integer
    Dim sData As Double, f As Double,
        fp As Double, x As Double, xt As Double

    n = Data.Rows.Count
    sData = WorksheetFunction.Sum(Data)

    x = sData / n / 3
    Do
        f = sData / x ^ 2 - (3 * n) / x
        fp = -2 * sData / x ^ 3 + 3 * n / x ^ 2
        xt = x
        x = x - f / fp
    Loop Until Abs(x - xt) / x < ErreurMax
    emvGamma = x
End Function
\end{Scode}
      \end{framed}
      \caption{Fonction VBA d'estimation du paramètre d'échelle d'une
        loi gamma par le maximum de vraisemblance.}
      \label{fig:resolution:emvGamma}
    \end{figure}
  \end{sol}

\end{exercice}

\begin{exercice}
  À l'aide de la méthode du point fixe, trouver le taux d'intérêt $i$
  tel que
  \begin{displaymath}
    a_{\angl{10}\,i}^{(12)} = \frac{1- (1 + i)^{-10}}{i^{(12)}} = 8,
  \end{displaymath}
  où
  \begin{displaymath}
    \left( 1 + \frac{i^{(12)}}{12} \right)^{12} = 1 + i.
  \end{displaymath}
  Comparer les résultats obtenus avec la méthode de Newton--Raphson.
%   \emph{Astuce}: Si vous avez une calculatrice TI-30XII, chacune des
%   itérations d'une méthode du type $x_{n} = g(x_{n-1})$ (point fixe,
%   Newton--Raphson, etc.) peut être exécutée en appuyant sur
%   \texttt{=}.  Il suffit de saisir la valeur de départ et d'appuyer
%   sur \texttt{=}. Ensuite, il faut saisir l'équation où $x$ devient
%   \texttt{Ans}. Chaque pression sur \texttt{=} exécutera ainsi une
%   itération de l'algorithme.
  \begin{sol}
    On cherche à résoudre l'équation
    \begin{displaymath}
      a_{\angl{10}\,i}^{(12)} = \frac{1- (1 + i)^{-10}}{i^{(12)}} = 8
    \end{displaymath}
    ou, de manière équivalente,
    \begin{displaymath}
      f(i) = \frac{1 - (1 + i)^{-10}}{12 (1 + i)^{1/12} - 1} - 8 = 0.
    \end{displaymath}
    Pour résoudre le problème à l'aide de la méthode du point fixe, on
    peut considérer la fonction
    \begin{displaymath}
      g_1(i) = i - f(i),
    \end{displaymath}
    mais, comme le graphique de la figure \ref{fig:resolution:a_angle:g1} le
    montre, la procédure itérative divergera. En posant
    \begin{displaymath}
      \frac{1 - (1 + i)^{-10}}{8} = 12[(1 + i)^{1/12} - 1],
    \end{displaymath}
    puis en isolant $i$ du côté droit de l'équation, on obtient une
    alternative
    \begin{displaymath}
      g_2(i) = \left( \frac{97 - (1 + i)^{-10}}{96} \right)^{12} - 1.
    \end{displaymath}
    Cette fonction satisfait les hypothèses pour l'existence et
    l'unicité d'un point fixe, mais sa pente est toutefois près de 1,
    donc la convergence sera lente; voir la figure
    \ref{fig:resolution:a_angle:g2}. Enfin, on peut considérer la fonction
    \begin{align*}
      g_3(i)
      &= i - \frac{f(i)}{f^\prime(i)} \\
      &= \frac{120(1 + i)^{-11}[(1 + i)^{1/12} - 1] - [1 - (1 +
        i)^{-10}](1 + i)^{-11/12}}{144 [(1 + i)^{1/12} - 1]^2}.
    \end{align*}
    On remarquera que cette dernière est la fonction utilisée dans la
    méthode de Newton--Raphson et sa pente est presque nulle, d'où une
    convergence très rapide; voir la figure
    \ref{fig:resolution:a_angle:g3}.
    \begin{figure}
      \begin{minipage}{0.32\linewidth}
        \centering
\includegraphics{resolution_equations-026}
        \subcaption{Fonction $g_1(x)$ \label{fig:resolution:a_angle:g1}}
      \end{minipage}
      \hfill
      \begin{minipage}{0.32\linewidth}
        \centering
\includegraphics{resolution_equations-027}
        \subcaption{Fonction $g_2(x)$ \label{fig:resolution:a_angle:g2}}
      \end{minipage}
      \hfill
      \begin{minipage}{0.32\linewidth}
        \centering
\includegraphics{resolution_equations-028}
        \subcaption{Fonction $g_3(x)$ \label{fig:resolution:a_angle:g3}}
      \end{minipage}
      \caption{Trois fonctions pour résoudre $a_{\angl{10}\, i}^{(12)}
        = 8$ par la méthode du point fixe.}
      \label{fig:resolution:a_angle}
    \end{figure}
    On obtient les résultats suivants:
\begin{Schunk}
\begin{Sinput}
> f <- function(i) (1 - (1 + i)^(-10))/(12 * ((1 + i)^(1/12) - 1)) - 8
> fp <- function(i) (120 * (1 + i)^(-11) * ((1 + i)^(1/12) - 1) - (1 - (1 + i)^(-10)) * (1 + i)^(-11/12))/(144 * ((1 + i)^(1/12) - 1)^2)
> g2 <- function(i) ((97 - (1 + i)^(-10))/96)^12 - 1
> g3 <- function(i) i - f(i)/fp(i)
> pointfixe(g2, start = 0.05)
\end{Sinput}
\begin{Soutput}
$fixed.point
[1] 0.047081

$nb.iter
[1] 40
\end{Soutput}
\begin{Sinput}
> pointfixe(g3, start = 0.05)
\end{Sinput}
\begin{Soutput}
$fixed.point
[1] 0.047081

$nb.iter
[1] 2
\end{Soutput}
\end{Schunk}
      \end{sol}
  \begin{rep}
    $0,0470806$
  \end{rep}
\end{exercice}

\Closesolutionfile{reponses}
\Closesolutionfile{solutions}

%%%
%%% Insérer les réponses
%%%
\input{reponses-resolution_equations}


%%% Local Variables:
%%% mode: latex
%%% TeX-master: exercices_methodes_numeriques
%%% End:
