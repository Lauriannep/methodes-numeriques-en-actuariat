\section{Exercices}
\label{sec:integration:exercices}

%%%
%%% Fichers de solutions et de réponses
%%%

\Opensolutionfile{reponses}[reponses-integration_numerique]
\Opensolutionfile{solutions}[solutions-integration_numerique]

\begin{Filesave}{reponses}
\bigskip
\section*{Réponses}

\end{Filesave}

\begin{Filesave}{solutions}
\section*{Chapitre \ref{chap:integration}}
\addcontentsline{toc}{section}{Chapitre \protect\ref{chap:integration}}

\end{Filesave}

%%%
%%% Début des exercices
%%%

\begin{exercice}
  À partir des formules d'approximation de $\int_{x_0}^{x_m} f(x)\,
  dx$, $m = 1, 2, 3$, développer les formules composées pour les
  méthodes d'approximation suivantes.
  \begin{enumerate}
  \item Trapèze ($m = 1$)
  \item Simpson ($m = 2$)
  \item Simpson 3/8 ($m = 3$)
  \end{enumerate}
  \begin{sol}
    Tel que vu à l'exemple~\ref{ex:integration:Lagrange}, le polynôme
    d'interpolation de Lagrange de second degré $P_2(x)$ est
    \begin{align*}
      P_2(x)
      &= f(x_0) \frac{(x - x_1)(x - x_2)}{(x_0 - x_1)(x_0 - x_2)}
      + f(x_1) \frac{(x - x_0)(x - x_2)}{(x_1 - x_0)(x_1 - x_2)} \\
      &\phantom{=}
      + f(x_2) \frac{(x - x_0)(x - x_1)}{(x_2 - x_0)(x_2 - x_1)}.
      &= \frac{1}{h^2} [f(x_0) (x - x_1)(x - x_2)
      + f(x_1) (x - x_0)(x - x_2) + f(x_2) (x - x_0)(x - x_1)]
    \end{align*}
  \end{sol}
\end{exercice}


\Closesolutionfile{reponses}
\Closesolutionfile{solutions}

%%%
%%% Insérer les réponses
%%%
\input{reponses-integration_numerique}

%%% Local Variables:
%%% mode: latex
%%% TeX-master: "methodes_numeriques-partie_3"
%%% coding: utf-8
%%% End:
