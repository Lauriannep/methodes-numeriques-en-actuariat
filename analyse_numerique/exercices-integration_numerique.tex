\section{Exercices}
\label{sec:integration:exercices}

%%%
%%% Fichers de solutions et de réponses
%%%

\Opensolutionfile{reponses}[reponses-integration_numerique]
\Opensolutionfile{solutions}[solutions-integration_numerique]

\begin{Filesave}{reponses}
\bigskip
\section*{Réponses}

\end{Filesave}

\begin{Filesave}{solutions}
\section*{Chapitre \ref*{chap:integration}}
\addcontentsline{toc}{section}{Chapitre \ref*{chap:integration}}

\end{Filesave}

%%%
%%% Début des exercices
%%%

\begin{exercice}
  À partir des formules d'approximation de $\int_{x_0}^{x_m} f(x)\,
  dx$, $m = 1, 2, 3$, développer les formules composées pour les
  méthodes d'approximation suivantes.
  \begin{enumerate}
  \item Trapèze ($m = 1$)
  \item Simpson ($m = 2$)
  \item Simpson 3/8 ($m = 3$)
  \end{enumerate}
  \begin{sol}
    On fait le cas de la méthode Simpson seulement, la procédure à
    suivre dans les autres cas étant tout à fait similaire.

    On souhaite développer la
    formule~\eqref{eq:integration:simpson:comp} à partir de
    l'approximation sur un sous intervalle donnée par
    l'équation~\eqref{eq:integration:simpson} pour l'intervalle $[x_0,
    x_2]$. De manière plus générale, nous avons
    \begin{equation*}
      \int_{x_{2j}}^{x_{2(j + 1)}} f(x)\, dx \approx
      \frac{h}{3} [f(x_{2j} + 4 f(x_{2j + 1}) + f_{2(j + 1)}], \quad
      j = 0, \dots, n - 1,
    \end{equation*}
    d'où
    \begin{align*}
      \int_a^b f(x)\, dx
      &\approx \sum_{j = 0}^{n - 1}
      \int_{x_{2j}}^{x_{2(j + 1)}} f(x)\, dx \\
      &= \frac{h}{3} \sum_{j = 0}^{n - 1}
      [f(x_{2j}) + 4 f(x_{2j + 1}) + f(x_{2(j + 1)})] \\
      &= \frac{h}{3}
      \left[
        f(x_0) +
        \sum_{j = 1}^{n - 1} f(x_{2j}) +
        4 \sum_{j = 0}^{n - 1} f(x_{2j + 1})
      \right. \\
      &\phantom{=}  + \left.
        \sum_{j = 0}^{n - 2} f(x_{2(j + 1)}) +
        f(x_{2n})
      \right] \\
      &= \frac{h}{3}
      \left[
        f(a) +
        2 \sum_{j = 1}^{n - 1} f(x_{2j}) +
        4 \sum_{j = 0}^{n - 1} f(x_{2j + 1}) +
        f(b)
      \right]
    \end{align*}
    puisque, par définition, $f(x_0) = f(a)$ et $f(x_{2n}) = f(b)$.
  \end{sol}
\end{exercice}

\begin{exercice}
  Évaluer numériquement les intégrales suivantes avec les méthodes du
  point milieu, du trapèze, de Simpson et de Simpson 3/8. Dans chaque
  cas, n'utiliser qu'un seul sous-intervalle, c'est-à-dire $n = 1$.
  \begin{enumerate}
  \item $\displaystyle\int_{0,5}^1 x^4\, dx$
  \item $\displaystyle\int_0^{0,5} \frac{2}{x - 4}\, dx$
  \item $\displaystyle\int_1^{1,5} x^2 \ln x\, dx$
  \item $\displaystyle\int_0^{\pi/4} e^{3x} \sin 2x\, dx$
  \end{enumerate}
  \begin{rep}
    \begin{enumerate}
    \item $0,158203$; $0,265625$; $0,194010$; $0,193866$
    \item $-0,266667$; $-0,267857$; $-0,267064$; $-0,267063$
    \item $0,174331$; $0,228074$; $0,192245$; $0,192253$
    \item $1,803915$; $4,143260$; $2,583696$; $2,585789$
    \end{enumerate}
  \end{rep}
  \begin{sol}
    Puisque $n = 1$, on peut utiliser directement les formules %
    \eqref{eq:integration:pointmilieu}, %
    \eqref{eq:integration:trapeze}, %
    \eqref{eq:integration:simpson} et %
    \eqref{eq:integration:simpson38}.
    \begin{enumerate}
    \item Point milieu:
      \begin{align*}
        \int_{0,5}^1 x^4\, dx &\approx
        2(0,25) (0,75)^4 = 0,158203 \\
        \intertext{Trapèze:}
        \int_{0,5}^1 x^4\, dx &\approx
        \frac{0,5}{2} [(0,5)^4 + 1] = 0,265625 \\
        \intertext{Simpson:}
        \int_{0,5}^1 x^4\, dx &\approx
        \frac{0,25}{3} [(0,5)^4 + 4(0,75)^4 + 1] = 0,194010 \\
        \intertext{Simpson 3/8:}
        \int_{0,5}^1 x^4\, dx &\approx
        \frac{3(0,5/3)}{8} [(0,5)^4 + 3(2/3)^4 + 3(5/6)^4 + 1] = 0,193866
      \end{align*}
    \item Point milieu:
      \begin{align*}
        \int_0^{0,5} \frac{2}{x - 4}\, dx &\approx
        2(0,25) (-2/3,75) = -0,266667 \\
        \intertext{Trapèze:}
        \int_0^{0,5} \frac{2}{x - 4}\, dx &\approx
        \frac{0,5}{2} [(-2/4) + (-2/3,5)] = -0.267857 \\
        \intertext{Simpson:}
        \int_0^{0,5} \frac{2}{x - 4}\, dx &\approx
        \frac{0,25}{3} [(-2/4) + 4(-2/3,75) + (-2/3,5)] \\
        &= -0,267064 \\
        \intertext{Simpson 3/8:}
        \int_0^{0,5} \frac{2}{x - 4}\, dx &\approx
        \frac{3(0,5/3)}{8} [(-2/4) + 3(-12/23) + 3(12/22) + (-2/3,5)] \\
        &= -0,267063
      \end{align*}
    \item Point milieu:
      \begin{align*}
        \int_1^{1,5} x^2 \ln x\, dx &\approx
        2(0,25) (1,75)^2 \ln 1,75 = 0,174331 \\
        \intertext{Trapèze:}
        \int_1^{1,5} x^2 \ln x\, dx &\approx
        \frac{0,5}{2} [0 + (1,5)^2 \ln 1,5] = 0,228074 \\
        \intertext{Simpson:}
        \int_1^{1,5} x^2 \ln x\, dx &\approx
        \frac{0,25}{3} [0 + 4 (1,25)^2 \ln 1,25 + (1,5)^2 \ln 1,5] \\
        &= 0,192245 \\
        \intertext{Simpson 3/8:}
        \int_1^{1,5} x^2 \ln x\, dx &\approx
        \frac{3(0,5/3)}{8} [0 + (7/6)^2 \ln (7/6) + (4/3)^2 \ln (4/3) \\
        &\phantom{\approx} + (1,5)^2 \ln 1,5] = 0,192253
      \end{align*}
    \item Point milieu:
      \begin{align*}
        \int_0^{\pi/4} e^{3x} \sin 2x\, dx &\approx
        2(\pi/8) e^{3\pi/8} \sin (\pi/4) = 1,803915 \\
        \intertext{Trapèze:}
        \int_0^{\pi/4} e^{3x} \sin 2x\, dx &\approx
        \frac{\pi/4}{2} [0 + e^{3\pi/4} \sin (\pi/2)] = 4,143260 \\
        \intertext{Simpson:}
        \int_0^{\pi/4} e^{3x} \sin 2x\, dx &\approx
        \frac{\pi/8}{3} [0 + 4 e^{3\pi/8} \sin (\pi/4) + e^{3\pi/4}
        \sin (\pi/2)] \\
        &= 2,583696 \\
        \intertext{Simpson 3/8:}
        \int_0^{\pi/4} e^{3x} \sin 2x\, dx &\approx
        \frac{3(\pi/12)}{8} [0 + 3 e^{3\pi/12} \sin (\pi/6) \\
        &\phantom{\approx} + 3 e^{3\pi/6} \sin (\pi/3) +
        e^{3\pi/4} \sin (\pi/2)] \\
        &= 2,585789
      \end{align*}
    \end{enumerate}
  \end{sol}
\end{exercice}

\begin{exercice}
  L'approximation de $\int_0^2 f(x)\, dx$ avec $n = 1$ donne $4$ avec la
  méthode du trapèze et $2$ avec la méthode de Simpson. Déterminer la
  valeur de $f(1)$.
  \begin{rep}
    $\frac{1}{2}$
  \end{rep}
  \begin{sol}
    Avec la méthode du trapèze et $n = 1$, on a $h = 2 - 0 = 2$ et
    l'approximation de l'intégrale est
    \begin{equation*}
      f(0) + f(2) = 4.
    \end{equation*}
    Avec la méthode de Simpson, $h = (2 - 0)/2 = 1$ et l'approximation est
    \begin{equation*}
      \frac{1}{3}[f(0) + 4 f(1) + f(2)] = 2.
    \end{equation*}
    En remplaçant la première équation dans la seconde et en
    résolvant, on trouve facilement $f(1) = 1/2$.
  \end{sol}
\end{exercice}

\Closesolutionfile{reponses}
\Closesolutionfile{solutions}

%%%
%%% Insérer les réponses
%%%
\input{reponses-integration_numerique}

%%% Local Variables:
%%% mode: noweb
%%% engine: xetex
%%% TeX-master: "methodes_numeriques-partie_3"
%%% coding: utf-8
%%% End:
