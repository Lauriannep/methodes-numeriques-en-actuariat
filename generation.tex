\chapter{Génération de nombres aléatoires}
\label{chap:generation}

\begin{objectifs}
\item
\end{objectifs}

Ce chapitre traite de la simulation de nombres (pseudo) aléatoires
distribués uniformément sur l'intervalle $(0, 1)$. La transformation
de ces nombres uniformes en nombres provenant d'autres distributions
statistiques fera l'objet du prochain chapitre.


\section{Pourquoi faire de la simulation?}
\label{sec:generation:pourquoi}

Les modèles mathématiques et la simulation stochastiques comportent
plusieurs avantages par rapport à l'expérimentation directe. Entre
autres, on note:
\begin{itemize}
\item la simulation est non destructrice et peu coûteuse;
\item le système considéré n'a pas besoin d'exister;
\item la simulation est facile à répéter;
\item l'évolution dans la simulation peut être plus rapide que dans la
  réalité;
\item la simulation permet de considérer des modèles très complexes
  impossibles à traiter analytiquement.
\end{itemize}
En revanche, au rayon des inconvénients:
\begin{itemize}
\item le coût (en temps et en argent) de modélisation et de
  programmation s'avère parfois important;
\item le temps d'exécution peut devenir excessif;
\item la simulation ne fournit que des estimations;
\item l'analyse statistique des résultats peut ne pas toujours être
  simple.
\end{itemize}

À la base, toute étude de simulation requiert une source de nombres
aléatoires. Or, ces nombres aléatoires ne sont pas toujours facile à
obtenir et la qualité de la source est primodiale pour que l'étude
soit fiable.



\section{Générateurs de nombres aléatoires}
\label{sec:generation:generateurs}

On veut obtenir des nombres issus d'une distribution $U(0, 1)$.
Ceux-ci peuvent par la suite être transformés en des nombres issus de
toute autre distribution. Quelques techniques possibles:
\begin{enumerate}
\item Utiliser les résultats de processus physiques aléatoires en
  apparence comme, par exemple,
  \begin{itemize}
  \item le lancer d'une pièce de monnaie ou d'un dé;
  \item des nombres pris au hasard dans des tableaux de rapports ou
    dans un annuaire;
  \item la roulette;
  \item le bruit électronique (tablaux RAND);
  \item les intervalles de temps dans un processus de décroissance
    radioactive sont considérés parfaitement aléatoires; voir
    \url{http://www.fourmilab.ch/hotbits/}.
  \end{itemize}
  L'utilisation de listes de nombres aléatoires est toutefois peu
  pratique avec un ordinateur, surtout si l'on a besoin de milliers ou
  de millions de nombres aléatoires.
\item Carrés centraux de von~Neumann: prendre un nombre à quatre
  chiffres, l'élever au carré puis extraire les quatre chiffres du
  milieu; recommencer. Par exemple:
  \begin{align*}
    8653^2 &= 74\undergroup{8744}09 \\
    8744^2 &= 76\undergroup{4575}36 \\
    4575^2 &= 20\undergroup{9306}25 \\
    &\vdots
  \end{align*}
\item Générateurs congruentiels et variations sur le même thème. Ce
  sont encore les plus utilisés encore aujourd'hui. Ils sont
  particulièrement utiles parce qu'aisément \emph{reproduisibles}.
\item Générateurs basés sur la suite de Fibonacci.
\item Générateurs chaotiques, etc.
\end{enumerate}

Dans la suite, nous nous concentrerons sur les générateurs de nombres
pseudo-aléatoires. En général, on exige d'un générateur de ce type
qu'il:
\begin{enumerate}
\item produise des nombres distribués approximativement uniformément;
\item produise des nombres approximativement indépendants dans un bon
  nombre de dimensions;
\item possède une période suffisamment longue (au moins $2^{60}$);
\item soit facilement reproduisible à partir d'un point de départ
  donné, mais qu'il soit autrement impossible à prédire.
\end{enumerate}


\subsection{Congruence et modulo}

Les générateurs congruentiels utilisent l'arithmétique modulo. Une
propriété de base de cette arithmétique est l'équivalence, ou
congruence, modulo $m$.

\begin{definition}
  Deux nombres $a$ et $b$ sont dits \emph{équivalents}, ou
  \emph{congruents}, modulo $m$ si la différence entre $a$ et $b$ est
  un entier divisible par $m$. Mathématiquement,
  \begin{displaymath}
    a \equiv b \bmod m \quad\Leftrightarrow\quad \frac{a - b}{m} = k, \quad
    k \in \mathbb{Z}.
  \end{displaymath}
\end{definition}

En d'autres mots, deux nombres sont équivalents modulo $m$ si la
distance entre ceux-ci est un multiple de $m$.

\begin{exemple}
  On a
  \begin{enumerate}
  \item $5 \equiv 14 \bmod 3$ car $\D \frac{14 - 5}{3} = 3$;
  \item $-1 \equiv 5 \bmod 3$ car $\D \frac{5 + 1}{3} = 2$;
  \item $0,33 \equiv 1,33 \bmod 1$; on notera que le calcul en modulo 1
    équivaut à retourner la partie fractionnaire d'un nombre;
  \item la minute dans l'heure est donnée en modulo 60: 00h15, 1h15,
    2h15,... sont des heures équivalentes modulo 60.
  \end{enumerate}
  \qed
\end{exemple}

De la définition de congruence découle celle de \emph{réduction
  modulo} ou \emph{résidu modulo}: si
\begin{itemize}
\item $a \equiv b \bmod m$ et
\item $0 \leq a < m$,
\end{itemize}
alors $a$ est le résidu de la division de $b$ par $m$, ou le résidu de
$b$ modulo $m$, c'est-à-dire
\begin{displaymath}
  a = b \bmod m \quad\Leftrightarrow\quad a = b - \left\lfloor
    \frac{b}{m} \right\rfloor m.
\end{displaymath}

La plupart des langages de programmation et logiciels à connotation
mathématique comportent un opérateur ou une fonction modulo:
\begin{itemize}
\item R: \verb|%%|;
\item Excel: \code{MOD()};
\item VBA: \verb|%|.
\end{itemize}


\subsection{Générateurs congruentiels linéaires}

Dans un générateur congruentiel linéaire, tout nombre dans la suite
générée détermine le nombre suivant par la formule
\begin{displaymath}
  x_i = (a x_{i - 1} + c) \bmod m,
\end{displaymath}
où $0 \leq x_i < m$ et
\begin{itemize}
\item $a$ est appelé le \emph{multiplicateur};
\item $c$ est appelé l'\emph{incrément};
\item $m$ est appelé le \emph{module};
\item $x_0$ (un nombre quelconque) est l'\emph{amorce} («\emph{seed}»).
\end{itemize}
De plus,
\begin{itemize}
\item $c = 0$ $\Rightarrow$ générateur \emph{multiplicatif};
\item $c \neq 0$ $\Rightarrow$ générateur \emph{mixte}.
\end{itemize}

Pour obtenir des nombre uniformes sur $[0, 1]$ ou $(0, 1)$, il suffit
de définir
\begin{displaymath}
  u_i = \frac{x_i}{m}.
\end{displaymath}

\begin{rems}
  \begin{enumerate}
  \item La méthode de génération des nombres est entièrement
    déterministe $\Rightarrow$ nombres \emph{pseudo}-aléatoires.
  \item Un générateur congruentiel est forcément périodique puisqu'il
    ne peut prendre, au mieux, que les valeurs
    \begin{itemize}
    \item $0, 1, 2, \dots, m - 1$ pour un générateur mixte;
    \item $1, 2, \dots, m - 1$ pour un générateur multiplicatif.
    \end{itemize}
  \item On cherche donc à avoir la période la plus longue possible
    tout en obtenant des suites en apparence aléatoires.
  \item Pour les générateurs multiplicatifs ($c = 0$), on atteint la
    période maximale $m - 1$ si
    \begin{itemize}
    \item $m$ est un nombre premier (on en choisira un grand);
    \item $a$ est une \emph{racine primitive} de $m$, c'est à dire que
      le plus petit entier $k$ satisfaisant
      \begin{displaymath}
        1 = a^k \bmod m
      \end{displaymath}
      est $k = m - 1$.
    \end{itemize}
    Valeurs populaires: $m = 2^{31} - 1$ (nombre premier de Mersenne),
    $a = 7^5$.
  \end{enumerate}
\end{rems}

\begin{exemple}
  Soit un générateur congruentiel multiplicatif avec $a = 7$ et $m =
  31$. Les cinq premiers nombres pseudo-aléatoires avec l'amorce $x_0
  = 19$ sont:
  \begin{align*}
    (7 \times 19) \bmod 31 &= 133 \bmod 31 = 9 \rightarrow x_1 \\
    (7 \times 9) \bmod 31 &= 63 \bmod 31 = 1 \rightarrow x_2 \\
    (7 \times 1) \bmod 31 &= 7 \bmod 31 = 7 \rightarrow x_3 \\
    (7 \times 7) \bmod 31 &= 49 \bmod 31 = 18 \rightarrow x_4 \\
    &\vdots
  \end{align*}
  \qed
\end{exemple}

\begin{exemple}
  \mbox{} \normalfont
  \lstinputlisting{exemple_2.3.R}
  \qed
\end{exemple}

\begin{exemple}
  \mbox{} \normalfont
  \lstinputlisting{exemple_2.4.R}
  \qed
\end{exemple}


\section{Générateurs utilisés dans Excel, VBA et R}

Avant d'utiliser pour quelque tâche moindrement importante un
générateur de nombres aléatoires inclus dans un logiciel, il importe
de s'assurer de la qualité de celui-ci. On trouvera en général
relativement facilement de l'information dans Internet.

Nous présentons ici, sans entrer dans les détails, les générateurs
utilisés dans Excel, VBA et R.

\subsection{Générateur de Excel}

La fonction à utiliser dans Microsoft Excel pour obtenir un nombre
aléatoire dans l'intervalle $(0, 1)$ est \code{ALEA()} (dans la
version française) ou \code{RAND()} (dans la version anglaise).

Dans les versions 2003 et 2007, Microsoft Excel utilise le générateur
de nombres aléatoire Whichmann--Hill. Ce générateur a longtemps été
considéré comme le meilleur disponible, mais a été supplanté ces
dernières années. Microsoft prétend que la période du générateur
Whichmann--Hill est $10^{13}$, mais omet alors de tenir compte de
littérature démontrant qu'elle est plutôt de $6,95 \times 10^{12}
\approx 2^{43}$, ce qui est aujourd'hui considéré trop court.

\begin{sloppypar}
  La mise en {\oe}uvre du générateur Whichmann--Hill dans Excel 2003
  avait le fâcheux défaut de pouvoir générer des nombres négatifs.
  Selon l'article 834520 de la base de connaissances Microsoft
  (\url{http://support.microsoft.com/kb/834520}), ce défaut a été
  corrigé dans le \emph{Service Pack} 1 de Office 2003, mais encore
  faut-il s'assurer que votre version de Excel est à jour. Sinon, une
  analyse basée sur des nombres aléatoires obtenus dans Excel pourrait
  s'avérer complètement erronée. La version 2007 du générateur est
  identique à la version 2003 corrigée
  (\url{http://support.microsoft.com/kb/828795}).
\end{sloppypar}

Consulter \cite{McCullough:Excel2007:2008} pour une discussion
détaillée de la génération de nombres aléatoires et d'autres
procédures statistiques dans Excel 2007, ainsi que les références
mentionnées dans cet article pour les version précédentes de Excel. De
plus, \citet{McCullough:MENTWH:2008} démontrent que le générateur de
Excel ne saurait être véritablement celui de Whichmann--Hill. Les
auteurs écrivent en conclusion:

\begin{quote}
  Twice Microsoft has attempted to implement the dozen lines of code
  that define the Wichmann and Hill (1982) RNG, and twice Microsoft
  has failed, apparently not using standard methods for verifying that
  an RNG has been correctly implemented. Consequently, users of
  Excel's "rand" function have been using random numbers from an
  unknown and undocumented RNG of unknown period that is not known to
  pass any standard tests of randomness.
\end{quote}


\subsection{Générateur de VBA}

\begin{sloppypar}
  La fonction \code{RND()} de VBA génère un nombre aléatoire. Selon
  l'article 231847 de la base de connaissances Microsoft
  (\url{http://support.microsoft.com/kb/231847}),
  le générateur de nombres aléatoires utilisé par la fonction
  \code{RND()} est un simple générateur congruentiel linéaire.
\end{sloppypar}

Étant donné l'avancement actuel des connaissances dans le domaine des
générateurs de nombres pseudo-aléatoires, l'utilisation d'un tel
générateur est tout-à-fait injustifiée et archaïque. De plus, le
générateur utilise toujours la même amorce et donc les suites de
nombres aléatoires sont toujours les mêmes. Pour toute
utilisation moindrement sérieuse de nombres aléatoires, il convient
donc d'éviter à tout prix la fonction \code{RND()} et de lui
préférer un appel à la fonction \code{RAND()} de Excel.


\subsection{Générateurs de R}

On obtient des nombres uniformes sur un intervalle quelconque avec la
fonction \code{runif} dans R. De plus, la fonction
\code{set.seed} permet de déterminer la valeur de l'amorce du
générateur aléatoire.

% Dans S-Plus, le générateur utilisé est appelé \emph{Super Duper}. Sa
% période est $2^{30} \times \nombre{4292868097} \approx 4,6 \times
% 10^{18}$.

R offre la possibilité de choisir entre six générateurs de
nombres aléatoires différents, ou encore de spécifier son propre
générateur. Par défaut, R utilise le générateur
Marsenne--Twister, considéré comme le plus avancé en ce moment.  La
période de ce générateur est $2^{\nombre{19937}} - 1$ (rien de moins!)
et la distribution des nombres est uniforme dans 623 dimensions
consécutives sur toute la période.

Pour de plus amples détails, consulter les rubriques d'aide des
fonctions \code{.Random.seed} et \code{set.seed}.

\input{exercices-generation}

%%% Local Variables:
%%% mode: latex
%%% TeX-master: "methodes_numeriques-partie_2"
%%% coding: utf-8
%%% End:
