\documentclass[letterpaper,11pt]{memoir}
  \usepackage{natbib,url}
  \usepackage[english,francais]{babel}
  \usepackage[autolanguage]{numprint}
  \usepackage{actuarialangle,amsmath,amsthm,icomma}
  \usepackage[noae]{Sweave}
  \usepackage{paralist}
  \usepackage[shortlabels]{enumitem}
  \usepackage{textpos}
  \usepackage{graphicx}
  \usepackage{manfnt}                  % \mantriangleright (puce)
  \usepackage{marvosym}                % \ForwardToEnd
  \usepackage{metalogo}                % \XeLaTeX logo
  \usepackage{applekeys}               % touches Mac
  \usepackage{fontawesome}             % \faExclamationSign, \faYouTubePlay
  \usepackage{answers,listings}
  \usepackage{pdfpages}                % couvertures
  \usepackage{xr}

  %%% ===================
  %%%  Style du document
  %%% ===================

  %% Polices de caractères
  \usepackage{fontspec}
  \usepackage{unicode-math}
  \defaultfontfeatures{Ligatures=TeX}
  \setmainfont[Scale=0.92,Numbers=OldStyle]{Lucida Bright OT}
  \setmathfont[Scale=0.92]{Lucida Bright Math OT}
  \setsansfont[Scale=1.0,Numbers=OldStyle]{Myriad Pro}
  \setmonofont[Scale=0.88]{Bitstream Vera Sans Mono}
  \newfontfamily\fullcaps[Letters=Uppercase,Numbers=Uppercase]{Myriad Pro}
  \usepackage{microtype}

  %% Définition de nouveaux symboles de la police Font Awesome pas
  %% encore définis dans le package fontawesome
  \def\faYouTubePlay{{\FA \symbol{"F16A}}}

  %% Couleurs
  \usepackage{xcolor}
  \definecolor{comments}{rgb}{0.7,0,0}
  \definecolor{shadecolor}{gray}{0}
  \definecolor{link}{rgb}{0,0.4,0.6}   % ~RoyalBlue de dvips
  \definecolor{url}{rgb}{0.6,0,0}      % rouge-brun
  \definecolor{citation}{rgb}{0,0.5,0} % vert foncé

  %% Hyperliens
  \usepackage{hyperref}
  \hypersetup{colorlinks, linktocpage,
    urlcolor=url, linkcolor=link, citecolor=citation,
    bookmarksopen, bookmarksnumbered, bookmarksdepth=subsubsection,
    pdfauthor={Vincent Goulet}}
  \setlength{\XeTeXLinkMargin}{1pt}
  \renewcommand{\appendixautorefname}{annexe}
  \renewcommand{\figureautorefname}{figure}
  \newcommand{\subfigureautorefname}{figure}
  \renewcommand{\tableautorefname}{tableau}
  \newcommand{\subtableautorefname}{tableau}
  \newcommand{\thmautorefname}{théorème}
  \newcommand{\exempleautorefname}{exemple}
  \newcommand{\definitionautorefname}{définition}
  \newcommand{\algorithmeautorefname}{algorithme}
  \newcommand{\exerciceautorefname}{exercice}

  %% Table des matières (inspirée de classicthesis.sty)
  \renewcommand{\cftchapterleader}{\hspace{1.5em}}
  \renewcommand{\cftchapterafterpnum}{\cftparfillskip}
  \renewcommand{\cftsectionleader}{\hspace{1.5em}}
  \renewcommand{\cftsectionafterpnum}{\cftparfillskip}

  %% Titres des chapitres
  \chapterstyle{hangnum}
  \renewcommand{\chaptitlefont}{\normalfont\Huge\sffamily\bfseries\raggedright}

  %% Marges, entêtes et pieds de page
  \setlength{\marginparsep}{7mm}
  \setlength{\marginparwidth}{13mm}
  \setlength{\headwidth}{\textwidth}
  \addtolength{\headwidth}{\marginparsep}
  \addtolength{\headwidth}{\marginparwidth}

  %% Titres des sections et sous-sections
  \setsecheadstyle{\normalfont\Large\sffamily\bfseries\raggedright}
  \setsubsecheadstyle{\normalfont\large\sffamily\bfseries\raggedright}
  \maxsecnumdepth{subsection}
  \setsecnumdepth{subsection}

  %% Listes. Paramétrage avec enumitem.
  \setlist[enumerate]{leftmargin=*,align=left}
  \setlist[enumerate,2]{label=\alph*),labelsep=*,leftmargin=1.5em}
  \setlist[enumerate,3]{label=\roman*),labelsep=*,leftmargin=1.5em,align=right}
  \setlist[itemize]{leftmargin=*,align=left}

  %% Options de babel
  \frenchbsetup{CompactItemize=false,%
    ThinSpaceInFrenchNumbers=true,
    ItemLabeli=\mantriangleright,
    ItemLabelii=\textendash}
  \addto\captionsfrench{\def\figurename{{\scshape Fig.}}}
  \addto\captionsfrench{\def\tablename{{\scshape Tab.}}}

  %%% =========================
  %%%  Nouveaux environnements
  %%% =========================

  %% Environnements d'exemples et al.
  \theoremstyle{plain}
  \newtheorem{algorithme}{Algorithme}[chapter]
  \newtheorem{thm}{Théorème}[chapter]

  \theoremstyle{definition}
  \newtheorem{exemple}{Exemple}[chapter]
  \newtheorem{definition}{Définition}[chapter]
  \newtheorem*{astuce}{Astuce}

  \theoremstyle{remark}
  \newtheorem*{remarque}{Remarque}
  \newtheorem*{remarques}{Remarques}
  \newenvironment{rem}{\begin{remarque} \mbox{}}{\end{remarque}}
  \newenvironment{rems}{\begin{remarques} \mbox{}}{\end{remarques}}

  %% Listes d'objectifs au début des chapitres
  \newenvironment{objectifs}{%
    \noindent
    \begin{framed}
      \vspace{-1.33\baselineskip}
      \begin{shaded}
        \noindent\sffamily\bfseries\textcolor{white}{Objectifs du chapitre}
      \end{shaded}
      \vspace{-0.6\baselineskip}
      \begin{itemize}[nosep]
        \small\sffamily}
      {\end{itemize}
    \end{framed}}

  %% Environnements de Sweave
  \DefineVerbatimEnvironment{Sinput}{Verbatim}{xleftmargin=\parindent}
  \DefineVerbatimEnvironment{Soutput}{Verbatim}{xleftmargin=\parindent}
  \DefineVerbatimEnvironment{Scode}{Verbatim}{xleftmargin=\parindent}
  \fvset{listparameters={\setlength{\topsep}{0pt}}}
  \renewenvironment{Schunk}{\vspace{\topsep}}{\vspace{\topsep}}

  %% Exercices et réponses
  \Newassociation{sol}{solution}{solutions}
  \Newassociation{rep}{reponse}{reponses}
  \newcounter{exercice}[chapter]
  \renewcommand{\theexercice}{\thechapter.\arabic{exercice}}
  \newenvironment{exercice}[1][]{%
    \begin{list}{}{%
        \refstepcounter{exercice}
        \ifthenelse{\equal{#1}{nosol}}{%
          \renewcommand{\makelabel}{\bfseries\theexercice}}{%
          \hypertarget{ex:\theexercice}{}
          \Writetofile{solutions}{\protect\hypertarget{sol:\theexercice}{}}
          \renewcommand{\makelabel}{%
            \bfseries\protect\hyperlink{sol:\theexercice}{\theexercice}}}
        \settowidth{\labelwidth}{\bfseries\theexercice}}
        \setlength{\leftmargin}{\labelwidth}
        \addtolength{\leftmargin}{\labelsep}
        \setlist[enumerate,1]{label=\alph*),labelsep=*,leftmargin=1.5em}
        \setlist[enumerate,2]{label=\roman*),labelsep=0.5em,align=right}}
      \item}
    {\end{list}}
  \renewenvironment{solution}[1]{%
    \begin{list}{}{%
        \renewcommand{\makelabel}{%
          \bfseries\protect\hyperlink{ex:#1}{#1}}
        \settowidth{\labelwidth}{\bfseries #1}
        \setlength{\leftmargin}{\labelwidth}
        \addtolength{\leftmargin}{\labelsep}
        \setlist[enumerate,1]{label=\alph*),labelsep=*,leftmargin=1.5em}
        \setlist[enumerate,2]{label=\roman*),labelsep=0.5em,align=right}}
      \item}
    {\end{list}}
  \renewenvironment{reponse}[1]{%
    \begin{enumerate}[label=\textbf{#1}]
      \item}
    {\end{enumerate}}

  %% Remarques importantes
  \newenvironment{important}{%
    \begin{framed}%
      \noindent
      \begin{minipage}{0.1\linewidth}
        \raisebox{-1.5em}[0em][0em]{\HUGE\faExclamationSign}
      \end{minipage}
      \begin{minipage}[t]{0.88\linewidth}}
      {\end{minipage}%
    \end{framed}}

  %% Informations
  \newenvironment{information}{%
    \begin{framed}%
      \noindent
      \begin{minipage}{0.1\linewidth}
        \raisebox{-1.5em}[0em][0em]{\HUGE\faInfoSign}
      \end{minipage}
      \begin{minipage}[t]{0.88\linewidth}}
      {\end{minipage}%
    \end{framed}}

  %% Remarques spécifiques OS X
  \newenvironment{osx}{%
    \begin{framed}%
      \noindent
      \begin{minipage}{0.1\linewidth}
        \raisebox{-1.5em}[0em][0em]{\HUGE\applekey}
      \end{minipage}
      \begin{minipage}[t]{0.88\linewidth}}
      {\end{minipage}%
    \end{framed}}

  %% Changements au fil de la lecture
  \newenvironment{gotoR}{%
    \begin{framed}%
      \noindent
      \begin{minipage}{0.07\linewidth}
        \raisebox{-0.5em}[0em][0em]{\LARGE\ForwardToEnd}
      \end{minipage}
      \begin{minipage}[t]{0.88\linewidth}\sffamily}
      {\end{minipage}%
    \end{framed}}

  %% Redéfinition de l'environnement de matrices de amsmath pour
  %% aligner les colonnes à droite. Pris dans
  %% <http://texblog.net/latex-archive/maths/matrix-align-left-right/>
  \makeatletter
  \renewcommand*\env@matrix[1][r]{\hskip -\arraycolsep
    \let\@ifnextchar\new@ifnextchar
    \array{*\c@MaxMatrixCols #1}}
  \makeatother

  %%% =====================
  %%%  Nouvelles commandes
  %%% =====================

  %% Noms de fonctions, code, etc.
  \newcommand{\code}[1]{\texttt{#1}}
  \newcommand{\pkg}[1]{\textbf{#1}}

  %% Indications de capsule vidéo
  \newcommand{\capsule}[2]{\href{#1}{#2}\marginpar{%
      \href{#1}{\raisebox{-0.5em}[0em][0em]{\HUGE\faYouTubePlay}}}}

  %% Raccourcis usuels vg
  \newcommand{\pt}{{\scriptscriptstyle \Sigma}}
  \newcommand{\abs}[1]{\lvert #1 \rvert}
  \newcommand{\norme}[1]{\lVert #1 \rVert}
  \newcommand{\mat}[1]{\mathbf{#1}}
  \newcommand{\diag}{\operatorname{diag}}
  \newcommand{\Esp}[1]{E\! \left[ #1 \right]}
  \newcommand{\esp}[1]{E [ #1 ]}
  \newcommand{\Var}[1]{\operatorname{Var}\! \left[ #1 \right]}
  \newcommand{\var}[1]{\operatorname{Var} [ #1 ]}
  \newcommand{\Prob}[1]{\operatorname{Pr}\! \left[ #1 \right]}
  \newcommand{\prob}[1]{\operatorname{Pr} [ #1 ]}
  \newcommand{\R}{\mathbb{R}}   % ensemble des réels

  %% Traitement du titre de partie
  \makeatletter
  \newcommand{\@parttitle}{}
  \newcommand{\parttitle}[1]{\renewcommand{\@parttitle}{#1}}
  \newcommand{\theparttitle}{\@parttitle}
  \makeatother

  %%% =========================
  %%%  Sections de code source
  %%% =========================
  \lstloadlanguages{R}
  \lstset{language=R,
    extendedchars=true,
    inputencoding=utf8/latin1,
    basicstyle=\small\ttfamily\NoAutoSpacing,
    commentstyle=\color{comments}\slshape,
    keywordstyle=\mdseries,
    showstringspaces=false}

  %%% =======
  %%%  Varia
  %%% =======

  %% Sous-tableaux et figures
  \newsubfloat{table}
  \newsubfloat{figure}

  %% Style de la bibliographie
  \bibliographystyle{francais}

  %% Aide pour la césure
  \hyphenation{%
    puis-que
    con-sole
    cons-tante
    nom-bre
    con-tenu
    con-naî-tre
    hexa-dé-ci-mal
    con-gru-en-tiels}

  %%% ======================================
  %%%  Page titre (sauf titre de la partie)
  %%% ======================================
  \renewcommand{\year}{2015}
  \title{%
    \bfseries\fontsize{42}{33}\selectfont {\fullcaps ACT 2002} \\
    \mdseries\fontsize{32}{33}\selectfont Méthodes numériques \\
                                          en actuariat}
  \author{%
    \bfseries\fontsize{16}{20}\selectfont Vincent Goulet \\
    \mdseries\fontsize{14}{18}\selectfont Professeur titulaire \textbar\
                                          École d'actuariat \textbar\
                                          Université Laval}
  \date{%
    \mdseries\fontsize{14}{18}\selectfont Notes de cours \textbar\
                                          Exercices \textemdash\
                                          édition \year}
